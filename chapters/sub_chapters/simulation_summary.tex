\section{Neutrinos, Backgrounds, and Routing}

A valid full reconstruction requires all packets to be collected regardless of neutrino type, energy within the valid range, interaction vertex, and should also be able to accept a range of incoming momentum angles.
In this final section of the chapter we combine the results of the neutrino interactions as well as the radiogenic and leakage backgrounds into the simulation framework described in Section~\ref{sec:tile_simulation}, which is based on the graph developed in Chapter~\ref{chap:qdb}.
This back-end simulation framework is also my sole and independent work.

\subsection{Combining the Digital and Physical Simulations}

To synthesize the results of the neutrino events and the background sources we take a reference 10 second window slice from a 1000 second simulation of backgrounds as described in Section~\ref{sec:simulating_tile}.
We then select a single $\nu_{e}$ interaction from the forward horn current direction, where we accept any incoming $\theta_{z}$, z-position, and energy.
This event is offset to occur at time: $\tau_{int} = $5.1 seconds.
We do not perform this readout simulations for all parameters in Table~\ref{table:neutrino_params} as Figure~\ref{fig:compare_integral_pdg} indicates that the local buffer requirement does not largely depend on neutrino flavor.

We choose $\tau_{int} = 5.1~\unit{s}$ so that the interaction occurs just after interrogation request from the aggregator node.
Although a slight effect, this ensures that the entire $\nu_{e}$ interaction is buffered onto all local FIFOs in the tile. 
We note that this time offset is the same for the push architecture, even though the neutrino reset packets will be sent when they are acquired.

\subsection{Remote FIFO Depth Results}

%%% Example of Digital Simulation Reconstruction
\begin{figure*}
  \centering
  \begin{subfigure}[b]{0.475\textwidth}
      \centering
      \includegraphics[width=\textwidth]{./images/mp60_16_slow_local_stack.pdf}
      \caption[Network2]%
      {{\small Network 1}}    
  \end{subfigure}
  \hfill
  \begin{subfigure}[b]{0.475\textwidth}  
      \centering 
      \includegraphics[width=\textwidth]{./images/mp60_16_slow_remote_stack.pdf}
      \caption[]%
      {{\small remote stack}}    
  \end{subfigure}
  \vskip\baselineskip
  \begin{subfigure}[b]{0.475\textwidth}   
      \centering 
      \includegraphics[width=\textwidth]{./images/mp60_16_slow_remote_transactions.pdf}
      \caption[]%
      {{\small transact}}    
  \end{subfigure}
  \hfill
  \begin{subfigure}[b]{0.475\textwidth}   
      \centering 
      \includegraphics[width=\textwidth]{./images/mp60_16_slow_route_fits.pdf}
      \caption[]%
      {{\small route fits}}    
  \end{subfigure}
  \caption[ Information on the 4 by 4 tile. ]
  {\small all of the data } 
  \label{fig:mp60_fast_plots_for_digital_sim}
\end{figure*}

%%% Example of Digital Simulation Remote Parameterization
\begin{figure*}
  \centering
  \begin{subfigure}[b]{0.475\textwidth}
      \centering
      \includegraphics[width=\textwidth]{./images/mp60_16_slow_route_fits.pdf}
      \caption[]%
      {\small 16 Sized Tile}    
  \end{subfigure}
  \hfill
  \begin{subfigure}[b]{0.475\textwidth}  
      \centering 
      \includegraphics[width=\textwidth]{./images/mp60_64_slow_route_fits.pdf}
      \caption[]%
      {\small 64 Sized tile}    
  \end{subfigure}
  \vskip\baselineskip
  \begin{subfigure}[b]{0.475\textwidth}   
      \centering 
      \includegraphics[width=\textwidth]{./images/mp60_140_slow_route_fits.pdf}
      \caption[]%
      {\small 140 Sized tile}    
  \end{subfigure}
  \hfill
  \begin{subfigure}[b]{0.475\textwidth}   
      \centering 
      \includegraphics[width=\textwidth]{./images/mp60_256_slow_route_fits.pdf}
      \caption[]%
      {\small 256 Sized Tile}    
  \end{subfigure}
  \caption[ Information on the 4 by 4 tile. ]
  {\small all of the data } 
  \label{fig:compare_slow_plots_for_digital_sim_slow}
\end{figure*}


%%% 4x4 Example Digital simulation results
\begin{figure*}
  \centering
  \begin{subfigure}[b]{0.475\textwidth}
      \centering
      \includegraphics[width=\textwidth]{./images/mp60_16_fast_local_stack.pdf}
      \caption[Network2]%
      {{\small Network 1}}    
  \end{subfigure}
  \hfill
  \begin{subfigure}[b]{0.475\textwidth}  
      \centering 
      \includegraphics[width=\textwidth]{./images/mp60_16_fast_remote_stack.pdf}
      \caption[]%
      {{\small remote stack}}    
  \end{subfigure}
  \vskip\baselineskip
  \begin{subfigure}[b]{0.475\textwidth}   
      \centering 
      \includegraphics[width=\textwidth]{./images/mp60_16_fast_remote_transactions.pdf}
      \caption[]%
      {{\small transact}}    
  \end{subfigure}
  \hfill
  \begin{subfigure}[b]{0.475\textwidth}   
      \centering 
      \includegraphics[width=\textwidth]{./images/mp60_16_fast_route_fits.pdf}
      \caption[]%
      {{\small route fits}}    
  \end{subfigure}
  \caption[ Information on the 4 by 4 tile. ]
  {\small example plots for a 4 $\times$ 4 tile. } 
  \label{fig:mp60_plots_for_digital_sim}
\end{figure*}

%%% Example of Digital Simulation Results for 16x16 tile
\begin{figure*}
  \centering
  \begin{subfigure}[b]{0.475\textwidth}
      \centering
      \includegraphics[width=\textwidth]{./images/mp60_16_fast_route_fits.pdf}
      \caption[]%
      {\small 16 Sized Tile}    
  \end{subfigure}
  \hfill
  \begin{subfigure}[b]{0.475\textwidth}  
      \centering 
      \includegraphics[width=\textwidth]{./images/mp60_64_fast_route_fits.pdf}
      \caption[]%
      {\small 64 Sized tile}    
  \end{subfigure}
  \vskip\baselineskip
  \begin{subfigure}[b]{0.475\textwidth}   
      \centering 
      \includegraphics[width=\textwidth]{./images/mp60_140_fast_route_fits.pdf}
      \caption[]%
      {\small 140 Sized tile}    
  \end{subfigure}
  \hfill
  \begin{subfigure}[b]{0.475\textwidth}   
      \centering 
      \includegraphics[width=\textwidth]{./images/mp60_256_fast_route_fits.pdf}
      \caption[]%
      {\small 256 Sized Tile}    
  \end{subfigure}
  \caption[ Information on the 16 by 16 tile. ]
  {\small 16 $\times$ 16 tile of results} 
  \label{fig:compare_fast_plots_for_digital_sim_fast}
\end{figure*}

%% table information
\begin{table}
	\begin{center}
		\begin{tabular}{|c|c|c|c|c|c|}
			\hline
			Freq. & Tile Size & Avg. Local Hits & Snake & Left & Trunk \\
			\hline
			5\% & 16 & 48.250 & 423.293 & 166.403 & 138.380 \\
			\hline
			0.5\% & 16 & 51.846 & 449.861 & 177.357 & 147.346 \\
			\hline
			5\% & 64 & 34.129 & 1332.440 & 286.929 & 227.595 \\
			\hline
			0.5\% & 64 & 36.268 & 1400.794 & 301.775 & 239.087 \\
			\hline
			5\% & 140 & 26.521 & 2298.912 & 355.037 & 262.448 \\
			\hline
			0.5\% & 140 & 28.173 & 2416.778 & 373.173 & 275.614 \\
			\hline
			5\% & 256 & 24.343 & 4020.649 & 465.629 & 354.405 \\
			\hline
			0.5\% & 256 & 25.752 & 4209.196 & 487.090 & 370.695 \\
			\hline
		\end{tabular}
	\end{center}
	\caption{Transaction Data}
	\label{tab:transact}
\end{table}

%% table information
\begin{table}
	\begin{center}
		\begin{tabular}{|c|c|c|l|r|l|r|l|r|}
			\hline
			Freq. & Tile Size & Local Hits & 95-S & 99-S & 95-L & 99-L & 95-T & 99-T \\
			\hline
			5\% & 16 & 939 & 320 & 1014 & 535 & 1736 & 607 & 1971 \\
			\hline
			0.5\% & 16 & 1014 & 322 & 975 & 603 & 1949 & 652 & 2125 \\
			\hline
			5\% & 64 & 1200 & 598 & 2191 & 1098 & 4394 & 975 & 4295 \\
			\hline
			0.5\% & 64 & 1307 & 403 & 1328 & 970 & 4298 & 974 & 4521 \\
			\hline
			5\% & 140 & 1182 & 852 & 3486 & 1455 & 6558 & 1343 & 6309 \\
			\hline
			0.5\% & 140 & 1393 & 440 & 1464 & 1327 & 6616 & 1382 & 6757 \\
			\hline
			5\% & 256 & 1456 & 1039 & 3637 & 2026 & 7679 & 2008 & 8250 \\
			\hline
			0.5\% & 256 & 1670 & 527 & 1668 & 1773 & 7460 & 1784 & 7368 \\
			\hline
		\end{tabular}
	\end{center}
	\caption{Buffer Data}
	\label{tab:buffers}
\end{table}

%% table information
\begin{table}
	\begin{center}
		\begin{tabular}{|l|l|c|c|c|c|}
			\hline
			Freq. & Tile Size & Snake Fit & Left Fit & Trunk Fit & Push Fit \\
			\hline
			5\% & 16 & 1.041 & 1.823 & 2.082 & 0 \\
			\hline
			0.5\% & 16 & 0.948 & 1.879 & 2.039 & 0 \\
			\hline
			5\% & 64 & 1.623 & 3.176 & 2.969 & 0 \\
			\hline
			0.5\% & 64 & 1.006 & 2.514 & 2.727 & 0 \\
			\hline
			5\% & 140 & 1.966 & 3.506 & 3.481 & 0 \\
			\hline
			0.5\% & 140 & 1.021 & 3.033 & 3.131 & 0 \\
			\hline
			5\% & 256 & 1.981 & 3.616 & 3.913 & 0 \\
			\hline
			0.5\% & 256 & 1.027 & 3.243 & 3.336 & 0 \\
			\hline
		\end{tabular}
	\end{center}
	\caption{Transaction Fit Results}
	\label{tab:fit}
\end{table}

\section{Summary and Further Studies}~\label{sec:further_studies}

\subsubsection{Future Neutrino Oscillation Studies}

Although an analysis of the reconstruction of the events involving oscillation are slightly beyond the scope of the work presented here, we provide a brief summary of how the work presented here helps in that future analysis.