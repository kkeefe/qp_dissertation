\section{Physical Simulation Studies}
%% TODO - channel level heatmap for neutrino event

We now discuss the methodologies of the physical simulations used as input to the digital tile simulation described in the previous section.

\subsection{Simulated Detector Properties}


%%
\subsection{Radiogenic Backgrounds as a Calibration Source}

In this section we comment on the ability to, but do not perform, the auto-calibration of the reset loss per charge.
The full calibration scheme is beyond the scope of the work presented here, with intracies that depend on the final implementation of the reset circuit as well as the replenishment circuit of the final Q-Pix ASIC.
Since this ASIC does not yet exist, the true verification of this methodology is not yet possible.
Nevertheless, we desribe the relevant portions to the problem here, since the charge calibration along with the frequnecy calibration and timestamp measurements determine Q-Pix's z-position reconstruction.

The main idea behind the auto-calibration of charge at the pixel level relies on using the known (and near constant) input current from the radiogenic backgrounds (mostly $^{39}$Ar) in the detetor. 
If a perfectly known and constant input current ($I_{o}$) source was input to a pixel it would produce a resets separated by a constant time ($\tau_{rtd}$).
It would be a straight foward matter to calculate the charge per reset: $Q_{o} = I_{o}*\tau_{rtd}$.

Then, to analyze the ability for an auto-calibration procedure for Q-Pix it is important to analyze and understand the long-running charge accumulation (resets) from backgrounds present in the LAr.
We use the following list of radiogenic sources of 10 separate iterations of runs of 1000 seconds.

\begin{table}
\begin{centering}
\begin{tabular}{|c c c c c c|}
 \hline
 Isotope & Rate [Bq/kg] & Region & Region Mass [kg] & Rate [Bq] & Number of Decays (per 10 s window) \\ [0.5ex]
 \hline\hline
  $^{210}$Po & 0.2 & PD [Bq/$m^2$] & 2.46856 & 0.493712 & 5 \\
  $^{60}$Co & 0.0455 & CPA & 90 & 4.095 & 41 \\
  $^{40}$K & 0.49 & APA & 258 & 1,264.2 & 12,642 \\
  $^{39}$Ar & 1.010 & bulk LAr & ~70,000 & 70,700 & 707,000 \\
  $^{42}$Ar & 0.000092 & bulk LAr & ~70,000 & 6.44 & 64 \\
  $^{42}$K  & 0.000092 & bulk LAr & ~70,000 & 6.44 & 64 \\
  $^{222}$Rn & 0.04 & bulk LAr & ~70,000 & 2800 & 28,000 \\
  $^{214}$Pb & 0.01 & bulk LAr & ~70,000 & 700 & 7,000 \\
  $^{214}$Bi & 0.01 & bulk LAr & ~70,000 & 700 & 7,000 \\
  $^{85}$Kr & 0.115 & bulk LAr & ~70,000 & 8050 & 80,500 \\
 \hline
\end{tabular}
\caption{The radiogenic background distribution is the same as that found in previous work~\citep{qpix:shion}.
For the 1000 s analysis the pre-rounded values are scaled up by a factor of 100 to achieve the correct normalization of events for each isotope.}
\end{centering}
\end{table}
~\label{table:radiogenic_backgrounds}

The well-known C++ based Geant4~\citep{geant4:AGOSTINELLI2003250} simulation toolkit is used to simulate particle decay and ionizing particle interactions within the LAr volume.
We use the energy deposited along the track from each ionizing particle with the W-value for liquid argon (23.6 eV) to determine the number of electrons deposited in the LAr.
The resulting number of electrons are then unifromly deposited over the individual track.
Then we calculate the probability of recombination for each electron following the "modified box" model~\citep{2013JInst...8P8005A}.

The time and location of drift for each electron is calculated with an applied transverse and longitudinal diffusion with values taken from~\ref{tab:lar_prop}.
The simulations for all particle interactions are run individually with randomly sampling some time interval within the 1000 second window.
All of the hits are then sorted by increasing time, so that the first hits read are the hits which occur the earliest.

This simulation procudes $\mathcal{O}(10^{11})$ hit interactions which produce $\mathcal{O}(10^{14})$ electrons, which in turn produces $\mathcal{O}(10^{9})$ resets.
To reduce the memory utilization of the simulation the electrons are accumulated on a hit-by-hit bases and are subdivided into a pre-determined 4$\times$4 cross-sectional area of the detector.
Each cross-sectional area defines a pixel.
The dimension of the LArTPC volume is 2.3$\unit{m}$ $\times$ 6.0$\unit{m}$ which divides into 575 pixels in the x-direction and 1500 pixels in the y-direction.
There are then a total of 862500 pixels, which store location and reset information.

As an example, the total resets from 1000 seconds of the simulation are shown in Figure~\ref{fig:background_simulation}.
The most active pixel receives $\approx 220$ resets during this simulation time.
The bins for the histogram are at the pixel level, and represnt the 4$\times$4$\unit{mm^{2}}$ of each pixel.

\begin{figure}[]
\centering
\includegraphics[width=\textwidth]{images/fullApaResets.png}
\caption{Full 1000 second of radiogenic background source simulation input into all pixels within the APA.
A close examination of the resets can reveal the additional resets from the backgrounds which are location dependent.
The majority of the uneven distribution of resets occur at the location of the APA bars due to the $^{40}$K isotope.
}
\end{figure}~\label{fig:background_simulation}

Additional information is tracked for each reset to identify the contribution from each radiogenic background to each reset.
A reset occurs when 6250 $e^{-}$ have accumulated on a pixel.
Any combination of the backgrounds can contribute some or all of the electrons required to produce a reset.
We refer to the contribution of the number of the electrons to the reset as the "weight" of the reset.
A distribution of all of the weights for all resets are shown in Figure~\ref{fig:pixel_truth_weight}.

\begin{figure}[]
\centering
\includegraphics[width=\textwidth]{images/pixel_reset_truth_track_id.pdf}
\caption{The total number of electrons contributed to all of the resets shown in Figure~\ref{fig:background_simulation}. 
The largest contribution, as expected, comes from the $^{39}$Ar source.
}
\end{figure}~\label{fig:pixel_track_id}

\begin{figure}[]
\centering
\includegraphics[width=\textwidth]{images/pixel_reset_truth_weight.pdf}
\caption{Truth weight of the individual resets for all combined resets in a 1000 s APA run.}
\end{figure}~\label{fig:pixel_truth_weight}


\begin{figure}[]
\centering
\includegraphics[width=\textwidth]{images/localHitsRadiogenic.pdf}
\caption{Baseline noise input for the digital simulation from all radiogenic sources and background leakage current.}
\end{figure}~\label{fig:reference_input_noise}

%% example RTD for a 16x16 tile
\begin{figure}
\centering
\begin{subfigure}{.5\textwidth}
  \centering
  \includegraphics[width=\textwidth]{images/radiogenicRTDtimescale.pdf}
  \caption{4096 Pixel Resets in log scale of time.}
\end{subfigure}%
\begin{subfigure}{.5\textwidth}
  \centering
  \includegraphics[width=\textwidth]{images/radiogenicRTDtimescale_stack_1d_noise.pdf}
  \caption{Flattend histogram against time}
\end{subfigure}
\caption{all 4096 pixels in a 16$\times$16 tile for 1000 seconds of radiogenic data. 
There are two clusters of resets on two different time scales.
The first cluster of resets near at $\tau_{rtd} \simeq \mathcal{O}(10^{-7})$ is due to events which produce more than one reset.
The second cluster of resets near at $\tau_{rtd} \simeq \mathcal{O}(10^{1})$ are from a "waiting" period where there is no charge buildup from decays and only background noise.
The dominant source of electrons on each pixel is from the $^{39}$Ar.
}
~\label{fig:radiogenic_rtd_timescales}
\end{figure}

\begin{figure}
\centering
\begin{subfigure}{.5\textwidth}
  \centering
  \includegraphics[width=\textwidth]{images/radiogenicRTDtimescale_1d.pdf}
  \caption{4096 Pixel Resets in log scale of time.}
\end{subfigure}%
\begin{subfigure}{.5\textwidth}
  \centering
  \includegraphics[width=\textwidth]{images/radiogenicRTDtimescale_stack_1d_noise.pdf}
  \caption{Flattend histogram against time}
\end{subfigure}
\caption{all 4096 pixels in a 16$\times$16 array for 1000 seconds of radiogenic data. 
There are two clusters of resets on two different time scales.}
~\label{fig:radiogenic_rtd_timescales_comparing_no_noise}
\end{figure}

sources of backgrounds are taken from~\citep{DUNE-FD_TDRv4:Abi_2020}
Radiological Backgrounds~\citep{ar39_backgrounds, phd_backgrounds}

%% list of sources here
%%

\subsection{Reset Distribution of Sources}

%% graphic for reset distribution from sources

%% graphic for energy deposited by source

%%%%%%%%%%%%%%%%%%%%%%%%%%%%%%%%%%%%%%%%%%%%%%%%%%%%%%%%%%%%%%%%%%%%%%%%%%%%%%%%
%%%%%%%%%%%%%%%%%%%%%%%%%%%%%%%   Part 2   %%%%%%%%%%%%%%%%%%%%%%%%%%%%%%%%%%%%%
%%%%%%%%%%%%%%%%%%%%%%%%%%%%%%%%%%%%%%%%%%%%%%%%%%%%%%%%%%%%%%%%%%%%%%%%%%%%%%%%
\section{Neutrino Beam High Energy Studies}~\label{sec:neutrino_studies}

Here we discuss the verification of the digital framework within the high energy regime.
For this we use as an input source neutrino events from the FNAL accelerate

\begin{figure}[]
\centering
\includegraphics[width=\textwidth]{images/dune_flux_energy_range.pdf}
\caption{Flux spectrum of neutrinos from the neutrino beam used in this study. This figure is taken from~\cite{electron_flux_image_2020}.}
\end{figure}~\label{fig:neutrino_flux}

%% TODO - Citation of the beam here

%% DUNE-FD TDR specification citation here
The DUNE-FD Vol.2 TDR~\citep{DUNE_FD_TDRv2_2020} describes in detail the design requirements for a future single-phase module.


%% fig example neutrino event

%% fig example neutrino pixelated event

%%

\subsection{Neutrino Event Parameters}

The parameters used to vary the tiles for these neutrino input energies are shown on Table~\ref{table:tile_params}.


\begin{table}
\begin{center}
\begin{tabular}{|| p{30mm} | p{30mm} | p{90mm} ||}
 \hline
 Name & Values & Relation \\ [0.5ex]
 \hline\hline
  Neutrino Energy & 0.25~GeV to 10~GeV, in steps of 0.25GeV & neutrino energy determines output secondary energy, which causes more resents and directly affects buffer depths. \\
 \hline
  Neutrino Type & $\nu_{e}$, $\bar{\nu_{e}}$, $\nu_{\mu}$, $\bar{\nu_{\mu}}$ & Oscillation measurements require a measurement of electron flavor neutrino appearance, and muon disappearance.\\
 \hline
  Horn Current & Forward and Reverse & Beam horn current selection affects which neutrino is present. \\
 \hline
  Z-position & 10\unit{cm}, 80\unit{cm}, 180\unit{cm}, 280\unit{cm}, 350\unit{cm}  & Interaction z-position above the anode plane. \\
 \hline
  Momentum Angle & 0\unit{\degree}, $\pm$2\unit{\degree}, $\pm$90\unit{\degree} & Different momentum angles are different Z-positions give larger track lengths within the active volume. \\
 \hline
\end{tabular}
\caption{The different neutrino simulation parameters which are passed into Geant4 based simulation.
  The original interaction products are created using GENIE~\citep{Andreopoulos:2009rq}.
  The output products from this generator are then configured using the different parameters described above.
}
\end{center}
\end{table}
~\label{table:neutrino_params}

\subsection{Neutrino Event Results}

% lego plot of all of the events

% image example th2i of a single event in the full APA

% image example of a full event with rotations
\begin{figure}
\centering
\begin{subfigure}{.5\textwidth}
  \centering
  \includegraphics[width=\textwidth]{images/example_zdir_scatter.pdf}
  \caption{Comparison of all integrals to current prototype depths.}
\end{subfigure}%
\begin{subfigure}{.5\textwidth}
  \centering
  \includegraphics[width=\textwidth]{images/example_xdir_scatter.pdf}
  \caption{Event Rotated about X-direction, or beam direction.}
\end{subfigure}
\caption{Same $\nu_{e}$ event in the middle direction of the APA but rotated upwards ($\theta_{z} = +90\deg$) and in the beam direction ($\theta_{z} = +0\deg$), respectively.}
\label{fig:compare_integral_nolabel}
\end{figure}
 
%% example integral of ASIC buffer depths with constant theta
\begin{figure}
\centering
\begin{subfigure}{.5\textwidth}
  \centering
  \includegraphics[width=\textwidth]{images/Const_Theta0_ASIC_stack_integral_pdg12_fhc.pdf}
  \caption{Stack of ASIC local FIFO depths}
\end{subfigure}%
\begin{subfigure}{.5\textwidth}
  \centering
  \includegraphics[width=\textwidth]{images/Const_Theta0_ASIC_integral_pdg12_fhc.pdf}
  \caption{Remote FIFO Depths}
\end{subfigure}
\caption{Events for different beam directions for $\nu_{e}$ particle for different momentum directions.}
\label{fig:example_asic_integral_value_constTheta}
\end{figure}


%% example integral of ASIC buffer depths with constant zpos value
\begin{figure}
\centering
\begin{subfigure}{.5\textwidth}
  \centering
  \includegraphics[width=\textwidth]{images/Const_Z180_ASIC_stack_integral_pdg12_fhc.pdf}
  \caption{Stack Of ASIC Buffer depths}
\end{subfigure}%
\begin{subfigure}{.5\textwidth}
  \centering
  \includegraphics[width=\textwidth]{images/Const_Z180_ASIC_integral_pdg12_fhc.pdf}
  \caption{Integral of ASIC local FIFO depths}
\end{subfigure}
\caption{Integral of the ASIC Buffers.}
\label{fig:example_asic_integral_value_constZpos}
\end{figure}

%% energy comparison for different different zpos and theta values
\begin{figure}
\centering
\begin{subfigure}{.5\textwidth}
  \centering
  \includegraphics[width=\textwidth]{images/Const_Theta0_ASIC_lepKE_multigraph_pdg12_fhc.pdf}
  \caption{Constant $\theta_{z} = 0$ direction.}
\end{subfigure}%
\begin{subfigure}{.5\textwidth}
  \centering
  \includegraphics[width=\textwidth]{images/Const_Z180_ASIC_lepKE_multigraph_pdg12_fhc.pdf}
  \caption{Constant Z-Position: $Z = 180\unit{cm}$.}
\end{subfigure}
\caption{Comparison of Buffer depths as a function of energy for different parameters of $\theta_{z}$ and z-position.}
\label{fig:example_asic_energy_comparison}
\end{figure}

%% comparing integral results here
%% 
\begin{figure}
\centering
\begin{subfigure}{.5\textwidth}
  \centering
  \includegraphics[width=\textwidth]{images/df_theta_cut.pdf}
  \caption{Colored by $\theta_{z}$ direction.}
\end{subfigure}%
\begin{subfigure}{.5\textwidth}
  \centering
  \includegraphics[width=\textwidth]{images/df_zpos_cut.pdf}
  \caption{Colored by Z-position}
\end{subfigure}
\caption{Comparison of Buffer depths as a function of energy for different parameters of $\theta_{z}$ and z-position.}
\label{fig:compare_integral_zpos_theta}
\end{figure}

\begin{figure}
\centering
\begin{subfigure}{.5\textwidth}
  \centering
  \includegraphics[width=\textwidth]{images/df_nolabel_line.pdf}
  \caption{Comparison of all integrals to current prototype depths.}
\end{subfigure}%
\begin{subfigure}{.5\textwidth}
  \centering
  \includegraphics[width=\textwidth]{images/df_horn_cut.pdf}
  \caption{Colored by Z-position}
\end{subfigure}
\caption{Comparison of Buffer depths as in incoming prototype and horn current direction from neutrino beam.}
\label{fig:compare_integral_nolabel}
\end{figure}