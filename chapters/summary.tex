\chapter{Summary and Outlook}~\label{chap:summary}

\section{Conclusions}

The results presented in this work provide the first tests and verification of the digital back-end for novel pixel readout technology targeting liquid Argon Time-Project-Chambers.

The first Q-Pix analog prototype using off-the-shelf analog components has been built and is currently taking measurements.
This prototype promises to provide gaseous argon diffusion measurements, which will likely be the first true physics measurement using a Q-Pix based readout.

We have built and verified the first digital prototype boards which have verified communication reliability to protect against potential data loss.
We developed a frequency calibration method for remote nodes to demonstrate Q-Pix's ability to have independent oscillators.
We used this prototype and verified the ability to reconstruct remote oscillator frequencies with a precision more than an order of magnitude required (0.1~\unit{ppm} < 1~\unit{ppm}).
These results are verified between two different interrogation frequencies of 0.1 Hz and 4 Hz.

We have developed several simulations to model the detector's response to long ($1000~\unit{second}$) run time exposure of radiogenic backgrounds as well as tested the ability to measure beam neutrino events at LBNF.
Our simulations show that the local (64) and remote (128) FIFO depths of the first prototype ASIC are too small to detect \unit{GeV}-scale neutrino events in a DUNE FD.
We estimate that the local FIFO depth should be at least be able to record 454 unique resets in order to fully capture 99\% of neutrino events with energy up to 10~\unit{GeV}.
These results are modified to a required 394 when accounted for expected (dis)appearance spectra given in~\citep{DUNE_FD_TDRv2_2020}
These results provide the first limit on the memory required for a Q-Pix ASIC if it is to be used in a DUNE-FD module to measure neutrino oscillations.

To test the remote FIFO depths and local oscillator frequency requirements we developed the first simulation to model the Q-Pix digital back-end response to physical events within a DUNE-FD APA.
We find that the distribution of the mean of the local oscillator frequencies needs to be $\approx 0.5\%$ in order to maintain obtain reliable remote FIFO depths with the current readout protocol. 
These results also indicate that the only reliable routing methodology is the "Snake" routing (Section~\ref{sec:snake_timing}), which is shown to be independent of both tile size and digital architecture (See Table~\ref{table:tile_params}).
The routing ("Snake") provides a unitary relationship between the local and remote FIFO depth requirements.

\subsection{The Future of Q-Pix}

The Q-Pix design is a novel readout technology.
It has been said, however, "novelty does not confer automatically benefit", David Nygren.
The full Q-Pix validation still awaits results to demonstrate its capabilities in a DUNE-FD module.
Namely, Q-Pix still needs to test both the analog and digital prototypes at cold liquid Argon temperatures.
Other tests are currently underway to verify the front-end analog capabilities of the Q-Pix readout.

The front-end requires a reliable replenishment circuit as well as low leakage current ($\approx 100~\unit{aA}$ or less) to be below radiogenic backgrounds.
The timing response of the replenishment circuit should be applied to the RTD results presented in this work.
Knowledge of the timing response of the analog front-end can be combined with the neutrino simulation events shown here to allow for accurate event reconstruction.
These reconstructed events will permit a complete analysis to estimate of Q-Pix's ability to perform neutrino oscillation measurements.

\subsection{Q-Pix's First and Second Digital Prototypes}

The work presented here can be accurately viewed as a means to understand the Q-Pix's first digital ASIC and as a guide toward the second digital design.
A key result of this work indicates that the local and remote FIFO depths of the second prototype should both be increased to at or above 394 to capture 99\% of neutrino oscillation interactions.
The reason the first prototype did not incorporate these larger buffer sizes was due to fabrication limitations of the ASIC.
If oscillator tests of the first prototype indicate that the mean drift between neighbor ASICs is reliably under 0.5\%, then the local oscillator would not need to be changed.
All other underlying logic, perhaps with the exception of First-Word-Fall-Through FIFOs, have been verified in the first digital prototype.
The packet communication tests presented here will be repeated on the ASIC prototypes to verify the control logic.

Eventually the Q-Pix front and back-end ASICs will likely be combined into a single chip.
Whether or not the Q-Pix prototypes will be combined during the second iteration remains to be seen.
Still, the motivation provided by the results presented here for the requirements of the digital portion of the second prototype remain unchanged.
