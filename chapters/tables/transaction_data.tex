\begin{table}
	\begin{center}
		\begin{tabular}{|c|c|c|c|c|c|}
			\hline
			Freq. & Tile Size & Mean Local Hits & Snake & Left & Trunk \\
			\hline
			5\% & 16 & 48.250 & 423.293 & 166.403 & 138.380 \\
			\hline
			0.5\% & 16 & 51.846 & 449.861 & 177.357 & 147.346 \\
			\hline
			5\% & 64 & 34.129 & 1332.440 & 286.929 & 227.595 \\
			\hline
			0.5\% & 64 & 36.268 & 1400.794 & 301.775 & 239.087 \\
			\hline
			5\% & 140 & 26.521 & 2298.912 & 355.037 & 262.448 \\
			\hline
			0.5\% & 140 & 28.173 & 2416.778 & 373.173 & 275.614 \\
			\hline
			5\% & 256 & 24.343 & 4020.649 & 465.629 & 354.405 \\
			\hline
			0.5\% & 256 & 25.752 & 4209.196 & 487.090 & 370.695 \\
			\hline
		\end{tabular}
	\end{center}
	\caption{Transaction summary data is shown.
	The mean local hits column indicates the mean average of resets injected into the ASICs within the tile from an electron neutrino events.
	The Snake, Left, and Trunk, columns indicate the mean number of remote packet transactions which occured during the full 10 second simulation run.
	As expected, the amount of packet transactions in the snake routing scales with the tile size, whereas the Left and Trunk routings do not.
	The frequency distribution of the tiles does not affect the total number of transactions in the simulated event.
	These results can be used to indicate the amount of power and active time required for a tile to fully readout an electron neutrino event. 
	For example, if a tile size of 256 with a snake routing takes 4020 packets on average to digitize the event, then there are a total of slightly more than one million packets sent.
	If the amount of power used during single packet transaction is known, this ratio could be used to estimate the dissipated power during the back-end readout. 
	}
	\label{tab:transact}
\end{table}