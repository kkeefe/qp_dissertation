\begin{table}
	\begin{center}
		\begin{tabular}{|c c c |c c|c c|c c|}
			\hline
			\multicolumn{3}{|c}{Routings} & \multicolumn{2}{c}{Snake} & \multicolumn{2}{c}{Left} & \multicolumn{2}{c|}{Trunk} \\
			\hline
			Tile Size & Frequency & Local Hits & 95\% & 99\% & 95\% & 99\% & 95\% & 99\% \\
			\hline
			16 & 5\% & 939 & 320 & 1014 & 535 & 1736 & 607 & 1971 \\
			 & 0.5\% & 1014 & 322 & 975 & 603 & 1949 & 652 & 2125 \\
			\hline
			64 & 5\% & 1200 & 598 & 2191 & 1098 & 4394 & 975 & 4295 \\
			& 0.5\% & 1307 & 403 & 1328 & 970 & 4298 & 974 & 4521 \\
			\hline
			140 & 5\% & 1182 & 852 & 3486 & 1455 & 6558 & 1343 & 6309 \\
			 & 0.5\% & 1393 & 440 & 1464 & 1327 & 6616 & 1382 & 6757 \\
			\hline
			256 & 5\% & 1456 & 1039 & 3637 & 2026 & 7679 & 2008 & 8250 \\
			 & 0.5\% & 1670 & 527 & 1668 & 1773 & 7460 & 1784 & 7368 \\
			\hline
		\end{tabular}
	\end{center}
	\caption{Remote buffer data summary.
	The tile sizes correspond to a square tile size except for 140, which corresponds to a 10$\times$14 tile.
	The frequency column represents the mean frequency distribution of each of the nodes within the tile.
	The values shown correspond to the amount of remote buffer depths required to fully transmit either 95\% or 99\% of events.
	Each column of data used more than 3000 electron neutrino events in the forward horn current direction.
	The local hits column indicates the average number of resets injected into the tile due to the electron neutrino event.
	The results also include the background radiogenic noise.
	}
	\label{tab:buffers}
\end{table}