\begin{table}
	\begin{center}
		\begin{tabular}{|c|c|c|c|}
			\hline
			FPGA Position & Mean (30~\unit{MHz}) & STD & $\frac{\delta f}{f_{o}}$*1e6 (ppm) \\
			\hline
			(0,0) & 245.543 & 2.379 & 0.079 \\
			\hline
			(0,1) & 190.646 & 2.979 & 0.099 \\
			\hline
			(0,2) & 153.908 & 3.334 & 0.111 \\
			\hline
			(0,3) & 248.831 & 3.843 & 0.128 \\
			\hline
			(1,0) & 192.729 & 2.860 & 0.095 \\
			\hline
			(1,1) & 210.905 & 3.405 & 0.114 \\
			\hline
			(1,2) & 116.212 & 3.984 & 0.133 \\
			\hline
			(1,3) & 159.824 & 4.158 & 0.139 \\
			\hline
			(2,0) & 351.431 & 3.685 & 0.123 \\
			\hline
			(2,1) & 193.845 & 4.285 & 0.143 \\
			\hline
			(2,2) & 200.278 & 4.071 & 0.136 \\
			\hline
			(2,3) & 152.633 & 4.263 & 0.142 \\
			\hline
			(3,0) & 183.359 & 3.954 & 0.132 \\
			\hline
			(3,1) & 209.788 & 4.561 & 0.152 \\
			\hline
			(3,2) & 192.277 & 4.169 & 0.139 \\
			\hline
			(3,3) & 171.302 & 4.538 & 0.151 \\
			\hline
		\end{tabular}
	\end{center}
	\caption{FPGA calibration results based on Hard Intterrogations at a frequnecy of 4~\unit{Hz}.
	The mean and standard deviation (STD) values are reconstructed for each ASIC within the tile as done in Figure~\ref{fig:frq_recon_node00} and ~\ref{fig:frq_recon_node33}.
	The listed STD value is the result of a gaussian fit performed on the adjusted frequencies.
	}
	\label{tab:fpga_calibration}
\end{table}