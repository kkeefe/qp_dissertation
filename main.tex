\documentclass[12pt]{uh_thesis}

\usepackage[hyphens]{url}
% \usepackage{lipsum}
% \usepackage{graphicx}
% \usepackage{pdflscape}
\usepackage{color}
% \usepackage{amsmath}
% \usepackage{amssymb}
\usepackage{longtable}
\usepackage{siunitx}
\usepackage{enumerate}
\usepackage{braket}
\usepackage{caption}
\usepackage{subcaption}
% \usepackage{tikz}
% \usepackage{pgf}
% \usetikzlibrary{arrows, automata, positioning}
% \usepackage{tikz-timing}[2014/10/29]
% \usetikztiminglibrary[rising arrows]{clockarrows}
%% Tentative: newtx for better-looking Times
\usepackage[T1]{fontenc}
\usepackage[utf8x]{inputenc}
\usepackage{newtxtext,newtxmath}

\usepackage[
    backend=bibtex,natbib,
    maxbibnames=15,
    maxcitenames=2,
    % IMPORTANT: load a style suitable for your discipline
    style=phys
]{biblatex}

\newsubfloat{figure}

% Reference a bus.
%
% Usage:
%
%     \busref[3::0]{C/BE}    ->   C/BE[3::0]
%     \busref*{AD}           ->   AD#
%     \busref*[3::0]{C/BE}   ->   C/BE[3::0]#
%
\NewDocumentCommand{\busref}{som}{\texttt{%
#3%
\IfValueTF{#2}{[#2]}{}%
\IfBooleanTF{#1}{\#}{}%
}}


\usepackage[colorlinks=true,linkcolor=blue,citecolor=red,urlcolor=blue]{hyperref}
\usepackage{cleveref}

\addbibresource{references.bib}

\begin{document}

\date{May 2023}

\title{Development of Digital Architectures for Pixelated Readout of Time Projection Chambers: Q-Pix}
\author{Kevin Keefe}

\degreeaward{Doctor of Philosophy}                 % Degree to be awarded
\department{Physics}
\university{Ke Kulanui o Hawai`i ma M\={a}noa}    % Institution name
\address{M\={a}noa, Hawai`i}                     % Institution address
\unilogo{manoa_seal.eps}                                 % Institution logo
\copyyear{2023}  % Year (of graduation) on diploma
\committee{Kurtis Nishimura 

Chairpersons\\ Jason Kumar, John Learned, Robert Wright, Peter Sadowski, Gary Varner}
\keywords{Field Programmable Gate Array, Time Projection Chamber, Neutrino Oscillation}

%% contact emails
%% kurtisn@phys.hawaii.edu; John G. Learned <jgl@phys.hawaii.edu>; Gary Varner <varner@phys.hawaii.edu>; Jason Kumar <jkumar@hawaii.edu>; Peter Sadowski <psadow@hawaii.edu>;

\orcid{0000-0001-5794-879X}

%% IMPORTANT: Select ONE of the rights statement below.
\rightsstatement{All rights reserved except where otherwise noted}
% \rightsstatement{Some rights reserved. This thesis is distributed under a [name license, e.g., ``Creative Commons Attribution-NonCommercial-ShareAlike License'']}

\maketitle[logo]

\begin{acknowledgements}

\end{acknowledgements}

\begin{abstract}
The Standard Model (SM) of physics has proven annoyingly successful in past decades, despite several measurements which hint at its incomplete description of nature.
The hunt for New Physics (NP) continues at higher energies ($>> 1 GeV$) with larger detectors ($\approx 10 kT$).
One such future detector is The Deep Underground Neutrino (DUNE) detector.
DUNE (as any beam detector) is a combination of two detectors, a near-detector (ND) and a far-detector (FD) for long-baseline neutrino oscillation measurements.
The DUNE FD will be a large scale ($\approx 40 kT$) Liquid Argon Time Projection Chabmer (LArTPC).
This 40-kT scale detector requires high precision in both timing ($<< \mu s$) and spatial resolutions ($\approx 1 mm$) for vertex reconstruction of interesting neutrino events.

This dissertation discusses recent progress and characterization of a novel implementation of a new pixelated LArTPC readout technology.
This novel readout is based on a pixel-level charge-integrate-reset circuit: Q-Pix.
We present the basic pixel-level readout circuit and the implications of such an implementation when applied at kiloton LArTPC scales.
We also show results from the first prototype implementation based on the Q-Pix readout, which was designed using only off-the-shelf electronics.
One problem with any pixelated readout is the ability to handle a large number of unique data channels, which in the case of the DUNE FD is $\approx 10^8$.
To address the scaling problem, we have developed and tested a modular digital back-end prototype as a proof of concept.
In this dissertation, we discuss nominal system requirements to achieve the DUNE-FD APA scale for radiogenic background sensitivity, as well as pixel-level calibration techniques for both timing and charge.
Simulations have also been performed based on projected radiogenic backgrounds and high-energy neutrino beamline events, providing initial estimates of the digital back-end requirements in both the quiescent and active states.
Finally, based on these results from the simulations and prototypes presented here, we discuss the nominal digital back-end readout constraints of a fully realized Q-Pix implementation for a DUNE-FD APA~.

% include incoming ASIC comments
\end{abstract}

\tableofcontents
\listoffigures
\listoftables
\printnomenclature

\mainmatter

%% chapter 1, introduce SM Physics, history of detectors leading to TPCs, DUNE, and how this relates
%% heading for this chapter
This chapter outlines and highlights useful background that will be explored in further detail in upcoming chapters as well as provides an outline for the thesis.
We begin with an introduction on the standard model, and how both its success and short comings drive larger and more expensive detectors at the intensity frontier.
To elucidate the issues at the forefront of the standard model we provide a brief history, with an emphasis on the detectors and experiments which lead to its formulation.
Next, we become more specific and discuss DUNE which is an example of a new, large, and expensive detector which aims to push beyond the Standard Model.
Finally, we relate the work presented in this dissertation is based on TPCs and how the novel readout design used here is suited for future expansion into larger detectors.

%% Where we are
\section{The State of Things: The Standard Model}

%% what is the SM: Summarize what it does well and what it doesn't do well
In the history of science, it is easily argued that the most successful of all models has been the Standard Model of physics.
The standard model was originally developed in mid to late 1970's, and is the model responsible for unifying the weak, strong, and electromagnetic forces together.
It has been made remarkable predictions about the existance of elusive neutrinos, and an extensive number of other particles.

%% elaborate more here on the standard model, what is it really?

Yet, despite its numerous achievements in predictive power and experimental verification we know today that it has crucial shortcomings. 
The Standard Model (SM) has no ability to account for Dark Matter or Dark Energy in the universe, nor the distribution (or the hierarchy) of neutrino masses, nor is it able to relate how gravity interacts with the other fundamental forces of nature (Unification).
It also doesn't account for some 'basic' properties it has, such as: why are there only three generations of laptonic particles (electron, muon, and tau)?
These short-comings offer hints for where to search for physics.
Physicists have known about these short comings from the conception of the Standard Model and have (to no avail) sought out what's next.

With a plethora of hints to search for NP, it can be useful to organize the efforts of search.
In 2008 the p5 committee did just this and labeled the three frontiers of physics as the cosmological, energy, and intensity frontiers.
Each of these frontiers offer different kinds of challenges and aim to search at the

%% cosmological frontier problems
The cosmological frontier aims to search for NP on extremely large time and distance scales by relying on observational techniques.
Cosmological measurements have shown the that majority of the universe's matter is not visible to light, and so we call it dark matter.
Additionally, the unverise is expanding at an accelerated rate, which we can tell from the blueshift of distance galaxies.
Likewise, cosmologists have also discovered that the universe is expanding due to some invisible energy in the universe, and so we call it dark energy.
The search for these dark causes of the universe lie within the realm of the cosmological frontier.

%% energy frontier problems
The energy frontier is concerned with the origin of mass.
The Large-Hadron Collider experiment is the archetypal experiment aimed at solving problems within this frontier.

%% intensity frontier problems (and the focus of this thesis)
The third (and final) frontier to mention is the Intensity frontier.
The Intensity Frontier of Physics (\citep{intensityfrontier2012_Hewett}) is one which requires very large and very precise measurements to gain the statistics to declare an observation.
In order to address the issues posed within this frontier the large scale detectors hunting for New Physics (NP) have continued to grow in size, energy sensitivity, and importantly cost: \citep{Juno:2022103927}.


%% how we got here
\section{How we got here.}

Many times since the early 20th century it was thought that the goal of physics was accomplished.
% Even during Max Plank's time (1858-1947) physics he was told (according to Planck himself) was nearly a complete and mature science, such a geometry.
However, during each of these moments of false triumph some new detector was built to take a new measurement; thus, the door to new understanding of nature is never shut.
This section provides a brief and (necessarily) incomplete history of significant measurements and detector developments relevant to particle physics.
In order to clear an obstacle, it is often helpful to remember the previous ones.


\subsection{A Century of new Detectors}

At the turn of the 20th century particle physics was in its infancy.
In 1900 Max Planck first introduces the concept of energy quanta for the first time concerning photons to eliminate the infamous ultra-violet catastrophe problem introduced by statistical mechanics.
JJ Thomson used a single cathode-ray tube to discover the electron and the nucleus, and won for himself the Nobel Pize in 1906.
Milikan's famous oil-drop experiment won him the Nobel Prize in 1923.

However, as each of these new discoveries solved problems only more questions were produced.
Once the nucleus was discovered to contain only protons and neutrons, the natural question arose: what holds all of the positive charge together in the center.
Thus, physicists cleverly named the new force which was stronger than the electromagnetic force: the Strong Force.

The bubble chamber was then invented in 1952 by Donald Glaser~\citep{bubbleChamber_PhysRev.87.665}.
These detectors proved significant in the discover of the W and Z bosons and ultimately allowed the unification of the electromagnetic and weak forces to form the electroweak theory.

Next the spark chamber eventually lead to the gradual development of the wire-spark chamber.
In 1968 Georges Charpak developed the Multi-Wire Proportional Chamber (MWPC) for which he (much later) won the Nobel Prize in 1992.
From this key insight a new detector concept was made possible.

\subsubsection{Time Projection Chambers}

Time Projection Chambers (TPC)~\citep{lartpc:nygren} have been shown to be extremely useful in high energy physics experiments due, in part, to their high resolution in both timing and spatial dimensions.
This detector was originally used in the Position-Electron Project PEP-4 experiment which measured electron-positron collisions from the 29 GeV electron beam produced at the Stanford Linear Accelerator (SLAC).
The first TPC design used high pressure gas and was able to measure 1000s of particle tracks per second (compared to 1-10) and provide full 3-D event reconstruction.


It did not take long for other experimentalists to generalize this concept to different elements or even to liquid.


\subsubsection{Noble Gases and Time Projection Chambers}

The technology of TPCs has greatly matured since their original inception.
in many kinds of detectors across HEP. TPCs can also incorporate two phases of a substance (liquid and gas), called Dual Phase (DP) TPCs.

the Xenon-1T is a dark matter experiment which is a dual-phase TPC \citep{Aprile_2017_xenon1T}.

The LUX experiment is a single phase TPC also hunting for dark matter.


A specific kind of TCP is a Liquid Argon Time Projection Chamber (LArTPC) \citep{rubbia1977liquid}.

%% include relevant LArTPCs here
recent work on LArTPCs (\citep{ArgoNeuT:PhysRevD.99.012002}, \citep{MicroBooNE:Acciarri_2017}, \citep{LArIAT:Acciarri_2020}).


Energy resolution of the LArTPCs within DUNE are still unknown to within a factor of 4 \citep{lartpc_energy_resolution:PhysRevD.99.036009}.



\subsection{An Escape: Catch the Neutrino}

More than 100 years ago in 1899 Ernest Rutherford observed beta decay.
Not too long afterward it was determined that the energy spectrum of the electrons resulting from this decay produced a spectrum.
This even lead some physicists to belief that perhaps the conservation of energy was violated.
However, the motivation to save this conservation law lead Wolfgang Pauli to the first prediction (1930) of the neutrino; the reason that the energy was a spectrum from the electron was that some of the energy was ``taken up'' by the neutrino.
Finally, some 26 years later in 1956 was the first observation of the electron neutrino~\citep{first_neutrino_measurement}.


After this first discovery is when the the answers, and mostly the questions started to pile in.


%% neutrino oscillation measurement here

%% solar neutrinos

%% neutrino mass oscillations
\begin{equation}
\begin{pmatrix}
\nu_e\\
\nu_{\mu}\\
\nu_{\tau}
\end{pmatrix}
=
\begin{pmatrix}
U_{e1}, U_{e2}, U_{e3} \\
U_{u1}, U_{u2}, U_{u3} \\
U_{\tau1}, U_{\tau2}, U_{\tau3}
\end{pmatrix}
\begin{pmatrix}
\nu_1\\
\nu_2\\
\nu_3
\end{pmatrix}
\end{equation}

%% what are the problems of neutrinos and which ones do we care about (electron / muon from a beam)
Super-K / SNO / KamLand / NOvA / daya bay / RENO / double chooz / t2k / minos

\citep{SNO_2002_neutrino_PhysRevLett.89.011301, neutrino_measurement_NOvA_2019_prl, t2k_2011_neutrino_PhysRevLett.107.041801}
\cite{reno_2012_neutrino_PhysRevLett.108.191802}
%\cite{minos_2006_neutrino_PhysRevLett.97.191801}
\citep{FUKUDA2002_solar_neutrino_oscillation}
\citep{kamland_2003_neutrino_PhysRevLett.90.021802}
\citep{daya_bay_2012_neutrino_PhysRevLett.108.171803}
\citep{doubleChooz_2012_neutrino_PhysRevLett.108.131801}

% Double Chooz used two identical gadolinium-doped liquid scintillator detectors
% t2k is (Tokai to Kamioka) is a long-baseline neutrino experiment, over 295 km
%% nd is at j-parc, far detector is at superk


%% physical interaction of neutrino scattering here


%% Highlight what DUNE is and its purpose
\section{The Deep Underground Neutrino Experiment}

The Deep Underground Neutrino Experiment (DUNE) is a long-baseline neutrino beam experiment \cite{DUNE_TDR_V1_Abi_2020, DUNE_FD_TDRv2_2020, DUNE_TDRv3_Abi_2020, DUNE-FD_TDRv4:Abi_2020}. 
DUNE is composed two detectors, a near (ND) and a far (FD) which are separated by a distance of 1300 km. 
The ND is located at Fermilab and its purpose is to characterize the source neutrino beam created there.
The FD is composed of four separate 10 kiloton modules, all of which will be a single-phase (SP) LArTPC based detector.
Two of these four modules at least will use a known wire-based readout technology and a vertical drift-readout.
The two remaining modules are considered modules of opportunity and their readout technology is yet unknown.
A purpose of this dissertation is show the viability of a novel readout technology.

\begin{figure}[]
\centering
\includegraphics[width=\textwidth]{images/LBNE_Graphic_061615_2016.jpg}
\caption{Simple Draw up of DUNE FD taken from \citep{dune_cdr_2016_arxiv}}
\end{figure}

DUNE has three main science goals, all of which are geared towards pushing beyond the standard model:
\begin{itemize}
    \item Hadron Decay
    \item Neutrinos from Core-collapse supernovae
    \item Beamline neutrino interactions.
\end{itemize}

We will discuss the relevance of each of these items, and in \ref{chap:qpix} we will further discuss how the work presented here relates to each of these topics. 


Conventional horizontal drift detection for foreseeable DUNE modules are already considered possible for lengths up to 6.5m \citep{DUNE_Vertical:Paulucci_2022}.


\subsection{Hadron Decay}
\label{sect:intro_decay}

Second generation proton decay studies in the ICARUS experiment: \citep{ICARUS_2001}.

%% why do we care about hadron decay

\subsection{Supernova Studies}
\label{sect:intro_supernova}

%% why do we care about supernova neutrinos, what will they tell us?
The principal decay chain follows the pattern:
\begin{equation}
    \nu_{e} + ^{40}Ar \rightarrow e^- + ^{40}Kr^*
\end{equation}

%% text from the cdr 
% The neutrinos from a core-collapse supernova are emitted in a burst of a few tens of seconds
% duration, with about half the signal emitted in the first second. The neutrino energies are mostly
% 4The lifetime shown here is divided by the branching fraction for this decay mode, $\tau$ /B, and as such is a partial
% lifetime.
% Volume 1: The LBNF and DUNE Projects LBNF/DUNE Conceptual Design Report
% Chapter 2: DUNE Science 2–18
% in the range 5–50 MeV, and the luminosity is divided roughly equally between the three known
% neutrino flavors. Current experiments are sensitive primarily to electron antineutrinos ($\nu_e$), with
% detection through the inverse-beta decay process on free protons5
% , which dominates the interaction
%rate in water and liquid-scintillator detectors. Liquid argon has a unique sensitivity to the electronneutrino (νe) component of the flux, via the absorption interaction on 40Ar,

% This interaction can be tagged via the coincidence of the emitted electron and the accompanying
% photon cascade from the 40K∗ de-excitation. About 3000 events would be expected in a 40−kt
% fiducial mass liquid argon detector for a supernova at a distance of 10 kpc. In the neutrino channel
% the oscillation features are in general more pronounced, since the νe spectrum is always significantly
%different from the νµ (ντ ) spectrum in the initial core-collapse stages, to a larger degree than is
%the case for the corresponding ν¯e spectrum. Detection of a large neutrino signal in DUNE would
% help provide critical information on key astrophysical phenomena such as
%  the neutronization burst,
%  formation of a black hole,
%  shock wave effects,
%  shock instability oscillations, and
%  turbulence effects.
% In addition to yielding unprecedented information on the mechanics of the supernova explosion,
% the observation of a core-collapse supernova in DUNE will also probe particle physics, providing
% neutrino oscillation signatures (with sensitivity to mass hierarchy and “collective effects” due to
% neutrino-neutrino interactions), as well as tests for new physics such as Goldstone bosons (e.g.,
% Majorons), neutrino magnetic moments, new gauge bosons (“dark photons”), “unparticles” and
% extra-dimensional gauge bosons

\subsection{Neutrino Oscillation}

Of all known particles the most elusive (hardest to detect and measure) is the neutrino.
For this reason the least is known about the neutrino.
What we do know about the neutrino is there are three pairs of them, associated with their leptonic partners: the electron, muon, and tau.

It came as a welcome shock that neutrino oscillation was first measured.
This oscillation indicates that a neutrino as it moves through space can change its state; a electron neutrino can oscillate into a muon neutrino or even a tau neutrino.
This happens because the mass eigenstate and flavor eigenstates which govern the neutrino are not equal.

%% why do we care about neutrino beam oscillations
\section{Towards Understanding}


%% what is the difference between NC and CC neutrino interactions.

\subsection{Pixelated Detectors in the Current Century}

Finally, in this last section we discuss recent progress of various detector technologies moving towards pixelization.
There are many motivating pressures for new detectors to adopt pixelated designs. 
Below we discuss two contributing factors: the development of electronics and computing algorithms.

First, previously pixelated detectors have historically been more difficult because of the issues of cost and size regarding the number of readout channels.
This is being addressed, in part, by the advent of newer, cheaper, and larger Field-Programmable-Gate Arrays (FPGAs).
One method for reducing the electronic overhead required in pixelated detectors is to use digital multiplexing.
Cheap, high channel FPGAs directly solve this problem. 
Other electronics development, such as the Silicon-Photomultiplier, offer much cheaper alternatives for large pixel counters compared to their historical counter-parts. 

%% antihydrogen
\citep{Sadowski_2017}
Another driving factor is the the development of Machine Learning (ML) algorithms, particularly Convectional Neural Network (CNN \citep{Sadowski2017DeepLI}). 
Recent industry has driven the need for CNNs to be able to correctly identify and label 2-D images of various kinds, and thus championed much of progress in this field and spawned many kinds of CNN algorithms. 
%% cite sadowski here
Recently, it has been shown how these kinds of algorithms extend into High Energy Physics (HEP) for particle identification.
A major issue at the Intensity Frontier of physics is the sheer amount of data to store and process. 
These ML algorithms provied a developed tool to automate the analysis of huge amounts of data ($>> 1 TB$) and have been shown to be quite accurate ($>99\%$) at particle identification in LArTPCs.

%% LArPix / Argon Cube
\subsection{Current Pixelization Efforts in TPCs}


%% goeldi inspiration here from LArPix

Additional work has been performed in recent years which show that LArTPCs can also utilized a pixel-based readout \citep{larpix:Dwyer_2018}, \citep{Asaadi_2018}.

%% SVSC
\subsection{Comments on Other applications of Pixelization}

\subsubsection{SANDD}

Another Example of a pixelated detector is \citep{SUTANTO2021_sandd_165409}.


\subsubsection{The Single Volume Scatter Camera}

This work is presented in greater detail in (Appendicies-\ref{chap:OS1}/\ref{chap:OS2}) and represents a substantial amount of my own individual contribution. 
I am the 2nd author on the the paper described in Appendix-\ref{chap:OS1} and the corresponding author of Appendix-\ref{chap:OS2}, where I also collected and analyzed all presented data therein.


%% chapter 2, the QPix Design concept
\chapter{A Novel Readout Technique for TPCs: Q-Pix}~\label{chap:qpix}

We begin this chapter in Section~\ref{sec:qpix_circuit} by introducing a novel pixel-based readout concept for TPCs.
Pixel-based readouts offer several advantages over the traditional wire readouts~\citep{lartpc_recon_problems_joshi_2015}.
A key improvement offered by pixelization is true 3-D image reconstruction.
This allows for sharper vertex reconstruction, which improves the overall resolution of a LArTPC and can reduce the time required for a NP measurement.
Other benefits include a reduction in data storage requirements and easier data analysis.

Next, in Section~\ref{sec:qpix_apa} we discuss how this readout technology can be extended to a DUNE-FD 10 kT module.
We compare the design specifications for the current wire-based readout of a DUNE-FD and discuss how the Q-Pix readout can meet similar design goals.
We note the advantages of pixelization come at the cost of increased design complexity.
The traditional wire-based readout of a DUNE-FD module will have hundreds to thousands of channels, while a pixel-based readout will have tens of millions of channels.
This number of channels is required to provide a stable readout over the expected lifetime of DUNE ($\approx 20$ years), where the electronics are operated continuously at liquid argon temperatures ($\simeq 87~\unit{K}$).
This dissertation addresses the channel size problem in Chapter~\ref{chap:sim}.

Two prototype ASICs have been developed to test each side of the readout shown in Figure~\ref{fig:qpix_frontandbackend}.
We refer to the ASIC designed to test the front-end as the "analog ASIC".
Likewise, we refer to the ASIC developed to test the back-end as the "digital ASIC".
The contributions to the design of the digital ASIC are a major part of this thesis.
In Section~\ref{sec:digital_back-end} we discuss the design challenges of a large pixelated detector, with a special emphasis on the requirements of the digital back-end.
Finally, in Sections~\ref{sec:qpix_photonics} and~\ref{sec:qpix_supernova} we briefly describe other work towards validating the Q-Pix design in a LArTPC.

\section{Q-Pix: The Circuit Level Design}
\label{sec:qpix_circuit}
The fundamental Q-Pix circuit~(\ref{fig:qpixCircuit}) was first introduced by Nygren and Mei~\citep{qpix:nygren:mei}.
The principle of the front-end circuit operates on producing a signal output from Schmitt trigger connected to a charge sensitive amplifier (CSA) circuit.
The input of the CSA is connected to the anode of a TPC, where drifted ionization charge accumulates.
Voltage is then built up across the capacitor from the accumulated charge according to the equation:

\begin{equation}~\label{eq:capacitor}
Q_{i} = C_{i}V_{i}
\end{equation}

When the capacitor voltage exceeds a set threshold ($V_{i}$), the Schmitt trigger is activated.
The Schmitt trigger output is sent as a signal input to the digital logic and recorded.
This signal is shown as the output in Figure~\ref{fig:qpixCircuit}.
The Schmitt trigger input to the digital back-end is recorded using a latch and a 32-bit counter incremented by a free running local oscillator ($f_{o} = 30~\unit{MHz}$).

\begin{figure}[]
\centering
\includegraphics[width=\textwidth]{images/qpix_circuit.jpg}
\caption{A simplified schematic of front-end Q-Pix Readout circuit.
The front-end is a charge sensitive amplifier (CSA) whose output is connected to a Schmitt trigger.
The trigger (output) occurs when enough charge as accumulated on the capacitor $C_{f}$ (currently 1~\unit{fC}).
One of the design parameters for the Q-Pix front-end is the choice of $C_{f}$.
This front-end circuit is designed within a single analog ASIC which has contains 16 of these circuits.
Image is taken from~\citep{qpix:nygren:mei}.}
\label{fig:qpixCircuit}
\end{figure}

The Q-Pix readout records 32-bit latch values as a response to the Schmitt trigger output; it does not measure voltage, charge, or time.
We refer to this 32-bit measurement as a "timestamp".
These timestamps are generated in response to the reset pulse sent from the Schmitt trigger.
Since these resets occur only in response to accumulated charge, the Q-Pix readout does not require an external trigger to acquire data.

There are two components of the Q-Pix readout: analog and digital.
The analog portion of the readout includes the CSA and the Schmitt trigger.
The digital portion of the readout is responsible for the timestamp record.
These two parts are shown in Figure~\ref{fig:qpix_frontandbackend}.

\begin{figure}[]
\centering
\includegraphics[width=\textwidth]{images/qpix_circuit_frontandbackend.jpg}
\caption{A modified image from~\citep{qpix:nygren:mei} is shown.
The blue box is the Q-Pix circuit~(Figure~\ref{fig:qpixCircuit}), which we refer to as the Q-Pix "analog front-end".
The right side of the image, encompassed in orange box, we refer to as the "digital back-end".
The back-end is responsible for providing the local oscillator as well as recording a reference counter to correspond to the time when the Schmitt trigger signal is received.
}
\label{fig:qpix_frontandbackend}
\end{figure}

\subsection{Current Reconstruction}
\label{sec:rtds_and_waveforms}
Here we describe the basic principle of reconstructing the input current from a collection of timestamp measurements.

A timestamp measurement indicates that a certain amount of charge has accumulated on the CSA.
Since total charge is conserved we can say that the total accumulated charge ($Q_{in}(t)$) is equal total amount of charge discharged from each reset ($Q_{out}(t)$) plus any residual charge still on the pixel ($Q_{c}(t)$).
Therefore, we can relate the total accumulated charge to the total charge discharged with the following equation:

\begin{equation}~\label{eq:qin}
Q_{in}(t) = Q_{out}(t) + Q_{c}(t)
\end{equation}

One important feature of the analog front-end is a replenishment circuit.
This circuit provides a constant current source which is used to remove a constant charge from the capacitor.
This constant charge removal greatly simplifies current reconstruction.
Assuming that each reset removes the same amount of charge ($Q_{o}$), we can rewrite the total charge out ($Q_{out}$) in terms of the integer number of resets at time t, ($N(t)$):
\begin{equation}~\label{eq:qout}
Q_{out}(t) = Q_{o} N(t)
\end{equation}

Equation~\ref{eq:qout} then gives us the measured current by definition ($I_{out} = \frac{dQ_{out}}{dt}$):
\begin{equation}~\label{eq:irecon}
I_{out} = \frac{d}{dt}(Q_{o}N(t)) = Q_{o}\frac{dN}{dt}
\end{equation}

We identify $\frac{dN}{dt}$ as the number of resets per unit time.
We can calculate the average current between two resets, or $\frac{dN}{dt} = \frac{1}{dT}$.
Then, $dT$ is the time difference between two resets and is measured by the local oscillator: $dT = \frac{N_{rtd}}{f_{o}}$, where $N_{rtd}$ is the difference between the two timestamps.
Equation~\ref{eq:irecon} becomes:

\begin{equation}~\label{eq:irecon_freq}
\boxed{I_{out} = \frac{Q_{o}f_{o}}{N_{rtd}}}
\end{equation}

Equation~\ref{eq:irecon_freq} shows the fundamental equation for the Q-Pix readout reconstruction.
There are three important parameters: the charge per reset, $Q_{o}$, the frequency of each local oscillator, $f_{o}$, and $N_{rtd}$ which is the difference between two timestamp measurements.
Examples of the transistor level simulations which use this current reconstruction are shown in Figures~\ref{fig:qpixRecon1} and~\ref{fig:qpixRecon2}.
Figures~\ref{fig:qpixRecon1} and~\ref{fig:qpixRecon2} show the charge reconstructions with a $Q_{o}$ value of 1~\unit{fC} and 0.3~\unit{fC}, respectively.

\begin{figure}[]
\centering
\includegraphics[width=\textwidth]{images/qpix_rtd_reconstruction_example.jpg}
\caption{Example current reconstruction using the Reset Time Difference (RTD) based on the Q-Pix readout design. 
Shown is transistor-level charge integration simulation results for minimum ionizing tracks in LAr.
The purple curve of the top panel represents charge accumulating on the CSA, where the green curve is the reset output from the Schmitt trigger.
The bottom panel shows input charge (purple curve) overlaid on top of the reconstructed current signal.
The $Q_{o}$ value taken here is equal to 1~\unit{fC}.
Image is taken from \citep{qpix:nygren:mei}.}
\label{fig:qpixRecon1}
\end{figure}

\begin{figure}[]
\centering
\includegraphics[width=\textwidth]{images/qpix_rtd_reconstruction_example_03fc.jpg}
\caption{Example current reconstruction using the Reset Time Difference (RTD) based on the Q-Pix readout design.
Shown is transistor-level charge integration simulation results for minimum ionizing tracks in LAr.
$Q_{o}$ was chosen to be $0.3~\unit{fC}$.
The bottom panel in this image shows a better match of the reconstructed signal with a smaller $Q_{o}$. 
However, more ($\approx$ 1.0~\unit{fC} / 0.3~\unit{fC}) resets are produced as a result. 
Image is taken from~\citep{qpix:nygren:mei}.}
\label{fig:qpixRecon2}
\end{figure}

Of the three parameters in Equation~\ref{eq:irecon_freq}, two of them need to be calibrated ($Q_{o}, f_{o}$).
Validation of the charge calibration is beyond the scope of this work, but is briefly described in Chapter~\ref{chap:sim} Section~\ref{sec:radiogenic_calib}.
The local oscillator frequency is a digital ASIC level calibration and its results are a product of this thesis.
It's procedure and first results are described in Chapter~\ref{chap:qdb}.
The remaining current reconstruction parameter is the timestamp, which is recorded by the digital back-end.

\subsection{Track Reconstruction}
One of the most important features of a TPC is the ability to accurately reconstruct the tracks of ionizing particles.
An intended use of a pixelated readout on any TPC is to show that there are improvements in the reconstruction of these 3-D track images.
Event reconstruction requires the three spatial coordinates (x,y,z) of the charge of a track as well as the time, t. 

The Q-Pix readout provides two coordinates (x and y) for "free".
Time is intrinsic to the Q-Pix datum and can also be provided by an external photonics system (see Section~\ref{sec:qpix_photonics}).
The first timestamp could also be used as the start time of any reconstructed track.
The final coordinate, z, is obtained with the LAr drift velocity (Table~\ref{tab:lar_prop}) and the drift time ($t_{\mathrm{drift}}$) by the equation:
\begin{equation}~\label{eq:driftDistance}
  z_{\mathrm{drift}} = v_{e}t_{\mathrm{drift}}
\end{equation}
The drift speed ($v_{e}$) is approximately constant whose value depends operating conditions of the TPC.
Equations~\ref{eq:irecon_freq} and~\ref{eq:driftDistance} together give the reconstructed average current and z-position values of the accumulated ionization charge on the pixel.

\subsection{Comments on Uncertainties}
Verification of the Q-Pix readout relies in part on tests to show that this timestamp data can be safely recorded and sent to disk without loss.
Here we briefly discuss potential uncertainties within the Q-Pix readout.

\subsubsection{Near Maximum Reset Rate}
Equation~\ref{eq:irecon_freq} relates the maximum current ($I_{max}$) in a Q-Pix readout at the limiting case of $N_{rtd} = 1$.
For $Q_{o} = 1~\unit{fC}$ and $f_{o} = 30~\unit{MHz}$, $I_{\mathrm{max}} = 30~\unit{nA}$.
However, since $N_{rtd}$ is a difference between two 32-bit timestamps, it can only take positive integer values.
This means that each current measurement can only take the following form, where $N$ is an integer: 
\begin{equation}~\label{eq:i_max}
I_{o} = \frac{I_{\mathrm{max}}}{N} \sim \frac{30~\unit{nA}}{N}
\end{equation}

Therefore, there can be large uncertainties in the measured currents with RTDs calculated with timestamps at frequencies near the frequency of the oscillator.
However, if there is sufficient total charge, these discrete uncertainties can be accounted for after digital processing.
An example of a periodic artificial input current of $I \approx I_{\mathrm{max}}/10$ is shown in Figure~\ref{fig:savgol} below.
The reconstructed charge over time is shown in Figure~\ref{fig:reconQ}.

\begin{figure}[]
\centering
\includegraphics[width=\textwidth]{images/savgol.pdf}
\caption{Arbitrary sine wave based current input.
The maximum amplitude is chosen to be close to $I_{\mathrm{max}}$.
Reset Charge is chosen to be 1 $\unit{fC}$ and digital clock frequency of 30 $\unit{MHz}$.
The input current amplitude is close to the maximum, so that the reconstructed current can is far from the actual.
This occurs since $N_{rtd}$ can only take integer values, the reconstructed current can only take values of $\frac{30}{N}$.
The red points indicate possible reconstructed current values, where the two highest points correspond to $\frac{30}{7}$ and $\frac{30}{8}$, respectively.
However, an example of a Savitzky–Golay (savgol) filter (dark blue) is performed on the resets after the fact with near agreement of the large input.
A use case of this kind of digital filtering would be applied to large current values only (peaks of the curves), and not for low current inputs where the pure timestamp difference provides better results.}
\label{fig:savgol}
\end{figure}

\begin{figure}[]
\centering
\begin{subfigure}{.5\textwidth}
  \centering
  \includegraphics[width=\textwidth]{images/reconQ.pdf}
  \caption{CDF}
\end{subfigure}%
\begin{subfigure}{.5\textwidth}
  \centering
  \includegraphics[width=\textwidth]{images/diffQ.pdf}
  \caption{Difference}
\end{subfigure}
\caption{Example charge reconstruction of Figure~\ref{fig:savgol} using the cumulative distribution function (CDF) of charge (a) and the error (b) in Coulombs as a function of time in~\unit{\mu s}.
When current is small there are no resets produced, as is seen in (a) by the periodic gaps in the resets.
The red line indicates the minimum charge required for a single reset.
The right plot indicates the difference between the true and reconstructed charge distributions compared to a single reset.
}
\label{fig:reconQ}
\end{figure}

\subsubsection{Comments on Reconstruction Requirements}
\label{sec:recon_uncert}
The uncertainties for the two transverse coordinates ($\hat{x}$ and $\hat{y}$) are related to the pixel's area.
If we assume that the electric field is uniform on average over all $\mathcal{O}(10^{7})$ pixels in an APA, then the charge drift will be uniformly distributed over the pixel.
The pixel size then determines the resolution: $\frac{4~\unit{mm}}{\sqrt{12}} \approx 1.15~\unit{mm}$.

To achieve the required uncertainty for the z-position measurement ($\sigma_{z} \simeq 1~\unit{mm}$), the uncertainty from $f_{o}$ must be small ($\sim$ 1~\unit{ppm}).
The 1~\unit{ppm} oscillator requirement is related to a time resolution of 1~\unit{\mu s} in Equation~\ref{eq:driftDistance}.
Other measurement uncertainties come from the Equation~\ref{eq:irecon_freq}.
The precision of the local oscillator frequency and the constant charge per reset determine the current reconstruction precision.
Variable replenishment circuits can greatly impact $Q_{o}$ from reset to reset, which will also affect $\sigma_{z}$.

\section{How Q-Pix fits into a 10 kt LArTPC}
\label{sec:qpix_apa}
A future target detector for the Q-Pix readout is a DUNE-FD 10 kt module.
To explore how Q-Pix could be used in this detector we provide a brief overview of the DUNE-FD electronics and compare with those requirements for Q-Pix based readout.
The simulation results presented in Chapter~\ref{chap:sim} are based on the detector volume of a single APA within a DUNE-FD module.

\subsection{The DUNE Far Detector Electronics}
The DUNE-FD is a modular assembly of Anode Plane Assemblies (APA) as shown in Figure~\ref{fig:dune_apa_tdr}.
Each Single-Phase (SP) 10 kT module will consist of 100-200 APAs (depending on the allowable drift distance).
The APA's full description can be found at~\citep{DUNE-FD_TDRv4:Abi_2020}.
Each APA stores its own the front-end electronics which are placed within the LAr and are shown in Figure~\ref{fig:dune_tpc_electronics}.

Figure~\ref{fig:dune_tpc_electronics} shows that each APA uses 20 FEMBs, each of which digitizes 128 of the total 2560 channels.
Of the 128 channels 40 are each taken from the U and V (induction) layers, and 48 wires are taken from the X (conduction) layer.
Each FEMB also houses a total of 18 ASICs which smooth, digitize, and aggregate data before being sent to the Warm Interface CRATE (WIC).
The total number of ASICs per APA is $18\times 20 = 360$.
Since each 10 kt module uses 150 APAs the total number of ASICs would be multiplied by 150.

\begin{figure}[]
\centering
\includegraphics[width=\textwidth]{images/dune_apa_motherboards.jpg}
\caption{Image taken from~\citep{DUNE-FD_TDRv4:Abi_2020}, Fig 1.12 of Section 1.8.
Image shows an overlay the the relevant charge collection wires within a DUNE-FD SP LArTPC.
}~\label{fig:dune_tpc_electronics}
\end{figure}

%% example image of DUNE-APA from DUNE-FD TDR.
\begin{figure}[]
\centering
\includegraphics[width=\textwidth]{images/dune_fd_tdr_apa_image.jpg}
\caption{Cross sectional picture of a single DUNE-FD APA bracket~\citep{DUNE-FD_TDRv4:Abi_2020}}
\label{fig:dune_apa_tdr}
\end{figure}

Each FEMB contains three different ASICs which are responsible for collecting the charge as it passes between the wires and sending it out of the cryostat.
The first ASIC is a waveform-shaping and amplification ASIC.
The second ASIC is the ADC ASIC and is responsible for the converting the analog signal to digital.
The final ASIC, called the COLDATA ASIC, merges the data streams from the previous ASICs and is responsible for communication between the motherboard and the outside world.

Table~\ref{tab:dune_tpc_elec} summarizes the electronic requirements expected for the DUNE-FD SP module.
The maximum expected data collection is expected to not exceed more than 30 PB/year, which corresponds roughly to $\approx 1 \mathrm{Gb/s}$ of continuous collection.
The expected wire electron noise level is design to be $\approx$ 1000 $e^{-}$.
Sampling frequency of 12 bit ADCs is 12 $\unit{MHz}$.
Large signals require a linear response of 500 k$e^{-}$, and ensures that fewer than 10\% of beam events experience saturation.
DUNE expects to draw less than 50 mW per channel, and incur less than 1\% dead channels.

\begin{table}
\begin{center}
\begin{tabular}{|| p{50mm} | p{40mm} | p{60mm} ||}
 \hline
 Description & Specification & Rationale \\ [0.5ex]
 \hline\hline
  System Noise & < 1000 $e^{-}$ & Provides >5:1 S/N on induction planes for pattern recognition and two-track separation. \\
 \hline
  Signal Saturation & 500,000 $e^{-}$ & Maintain calorimetric performance for multi-proton final state. \\
 \hline
  Cold Electronics Power Consumption & < 50~\unit{mW} per channel & No bubbles in LAr to reduce HV discharge risk\\
 \hline
  Number of Channels per front-end motherboard & 128 & The total number of wires on one side of an APA, 1,280, must be an integer multiple of the number of channels on the FEMBs. \\
 \hline
  Dead Channels & < 1\% & Minimize the degradation in physics performance over the > 20-year detector operation. \\
 \hline
  Maximum diameter of conduit enclosing the cold cables while they are routed through the APA frame. & 6.35 cm (2.5") & Avoid the need for further changes to the APA frame and for routing the cables along the cryostat walls \\
 \hline
\end{tabular}
\caption{Selected requirements of DUNE-FD TPC electronics and expected Q-Pix Design goals of first generation ASIC development for comparison.
The table information is taken from the inner three columns of Table 4.1 in~\citep{DUNE-FD_TDRv4:Abi_2020}.
Due to the different charge sensitive geometries between a wire and pixel-based readout, the required noise and number of channels are not easily comparable.
}
\label{tab:dune_tpc_elec}
\end{center}
\end{table}

\subsection{Q-Pix Comparison to DUNE-FD Electronics}
Each APA is 6.324 $\unit{m}$ $\times$ 2.316 $\unit{m}$, for a total area of 14.646 $\unit{m^{2}}$.
From these dimensions the expected channel count of the Q-Pix readout on the DUNE-FD APA is
\begin{equation}
  N_{\mathrm{pix}} = 14.646 m^{2} * \frac{1 \mathrm{pixel}}{4 mm^{2}} * \frac{ 1000^{2} mm^{2} }{m^{2}} = 915399
\end{equation}

The total number of free running oscillators ($N_{osc}$) per DUNE-APA for a given pixel pixel of $4~mm^{2}$ is:
\begin{equation}~\label{eq:nosc}
N_{\mathrm{osc}} = \frac{915399}{16} \approx 57213
\end{equation}

$N_{\mathrm{osc}}$ represents the total number of front-end ASICs whose data must be aggregated and sent outside of the cold electronics to a warm interface.
Therefore we expect the order of the number of free running oscillators per DUNE-APA $\mathcal{O}$($10^5$).
This also gives an order of magnitude estimate of the increase of number of ASICs compared to the MWPC readout of Single-Phase (SP) DUNE-FD.

To have a comparable power consumption to DUNE-FD, which has 2560 channels, Q-Pix would need less than $\approx 140~\mathrm{\mu W}$ of power consumption per channel.
Too much power dissipated in the LAr creates bubbles, which is a high voltage (HV) discharge risk.
The total number of channels for a 10 kT module is based on 150 DUNE-APAs or $2560\times 150 = 384000$.
Thus, the number of additional analog channels that Q-Pix must measure compared to a typical wire readout increases by a factor of $915399 / 2560 \approx 357$.

Q-Pix will instead offer conversion from analog (charge) to digital (32 bit time) signals on a single ASIC.
These front-end ASICs would be arrayed in modular tiles within a single APA.
Where the tiles themselves would be connected and spread out to cover the entire area of an APA.

To connect the entire APA the Q-Pix readout will design modular tiles which will hold a subset of nodes ($\sim$ 16$\times$16 ASICs).
Each tile will interface with a single FPGA (or other ASIC) chip which would concentrate the digital data for each tile; we refer to this FPGA as the DAQ-Node (DN).
Then, each DN can interface could optionally connect a single concentrator FPGA for the entire APA.
This final concentrator would send the data to the Warm-Interface-Cards (WIC) out of the cold electronics (CE).

\begin{table}
\begin{center}
\begin{tabular}{|| p{40mm} | p{40mm} | p{70mm} ||}
 \hline
 Description & Specification & Rationale \\ [0.5ex]
 \hline\hline
  System Noise & $\approx 300 e^{-}$ & Provides $\approx$ 17:1 S/N ratio, a component of front-end integrator. \\
 \hline
  Signal Saturation & 30~\unit{nA} per pixel & Upper limit from local oscillator frequency and integrator reset. \\
 \hline
  Cold Electronics Power Consumption & $< 100 \unit{\mu W}$ per channel & Equivalent power consumption for heating found in DUNE-FD. \\
 \hline
  Number of Channels per Tile & 4096 & Design parameter to be calculated. \\
 \hline
\end{tabular}
\caption{Q-Pix based requirements that are compared to the equivalent DUNE-FD SP module found in Table~\ref{tab:dune_tpc_elec}.
Results here are necessarily speculative, but provide an insight to the design goal.
The system noise is a contribution of leakage current per reset, uneven electric fields in the TPC, or uncertainties in the replenishment circuit per reset.
Signal saturation is defined as the maximum measurable current based $Q_{o} = 1~\unit{fC}$ and $f_{o} = 30~\unit{MHz}$.
The Q-Pix readout can also be saturated by too many resets if the FIFO buffers overflow (see Chapter~\ref{chap:sim} for further discussion).
The power consumption per channel is a direct division compared to Table~\ref{tab:dune_tpc_elec}.
The number of channels per tile depends on the tile size as well as the number of pixels per ASIC (see Chapter~\ref{chap:qdb} for further discussion).
}
\label{tab:qpix_tpc_elec}
\end{center}
\end{table}

\section{The Digital Back-end}
\label{sec:digital_back-end}
We define the digital back-end as the part of the larger Q-Pix readout system that is responsible for handling the timestamp data once it is recorded.
This sub-system must be able to record and store data, be robust against SPF, define error states, and more.

In this section we introduce considerations which guided the design of the first two Q-Pix ASIC prototypes.
These design choices for the digital prototype are enumerated in Table~\ref{table:digital_design_params}.

\subsection{Digital ASIC Prototype Design Choices}
\label{sec:design_choices}
Table~\ref{table:digital_design_params} highlights some design choices of the first digital Q-Pix ASIC prototype.
Four neighbor connections were chosen as the design choice two allow for communication in either direction along the x and y axes.
The local and remote FIFO depths determine how many resets and communication packets the ASIC may store at any one time.

\begin{table}
\begin{center}
\begin{tabular}{||p{30mm} | p{30mm} | p{90mm}||}
 \hline
 Parameter Name & Value & Description \\ 
 \hline\hline
Local Oscillator Frequency & 30~\unit{MHz} & Determines maximum current reconstruction (Equation~\ref{eq:irecon_freq}). Stability of local oscillator also determines z-position reconstruction uncertainty (Sect.~\ref{sec:recon_uncert}). \\
 \hline
Connections & 4$\times$2 (Tx and Rx) &  Eight differential pair connections are made to support four transmitter (Tx) and receiver (Rx) lines. \\
 \hline
Communication Protocol & Endeavor & Protocol determines packet stability based on oscillator frequency as well as packet transaction time. Results of these tests are done in Chapter~\ref{chap:qdb}. \\
 \hline
Timestamp Bits & 32 & Determines the total number of unique counts each timestamp value can take. Also determines the "wrap-around" time based on the local oscillator.  \\
 \hline
Local\newline FIFO Depth & 64 & Total number of timestamps the ASIC can store before running out of memory. ASIC will not record additional resets until emptied. \\
 \hline
Remote\newline FIFO Depth & 128 & Total number of remote packets ASIC can store before running out of memory. ASIC will not write additional packets from neighbors. \\
 \hline
 \hline
\end{tabular}
\caption{Summary of design parameters for the first Q-Pix digital ASIC.
The local oscillator is a ring oscillator with a targeted mean of 30~\unit{MHz}.
Each ASIC will be able to communicate with up to four neighbor nodes via a custom "Endeavor" protocol.
The testing and verification of the endeavor protocol is found in Chapter~\ref{chap:qdb}.
The values of the local and remote FIFO depth were selected due to fabrication requirements, whose future values are discussion of Chapter~\ref{chap:sim}.
}
\label{table:digital_design_params}
\end{center}
\end{table}

\subsection{Single Point Failures}
Q-Pix digital design should provide ``robust resilience'' against single point failure (SPF).
The readout technology presented here relies on huge numbers of readout channels ($10^{8}$) compared to current MWPC designs ($10^{5}$).
As such, extra care must be made in designing new technology to improve over established, seemingly simpler means.

This principal guides design choices such as the use of independent local oscillators at the pixel-level instead of a provided distributed clock.
This design choice, in particular, is discussed at length in Chapter~\ref{chap:qdb}, and the findings presented there are one of the major contributions presented in this thesis.

This thesis focuses on a design that avoids SPF and describes a continual time calibration of each local oscillator that meets the requirements presented here.
However, the amount of data produced depends on the charge collected in each event, therefore the amount of data collected can not be known before recording.

\section{Q-Pix and Light Detection}
\label{sec:qpix_photonics}
The results of this thesis analyze the response of the Q-Pix digital readout without any analysis paid towards photon collection.
However, recent progress regarding amorphous selenium (aSe)~\citep{https://doi.org/10.48550/arxiv.2207.11127} has been made towards inclusion of an optical system.

The current pixel dimensions of Q-Pix are 4~$\unit{mm}$ $\times$ 4~$\unit{mm} $ which have a total active area of 16~$\unit{mm^{2}}$.
Most of this active area is unused for the charge collection pad, which could be as small as drill-hole via (6 mil $<<$ 16~\unit{mm^2}).
Most of the remaining area, then, could be plated with a photo-sensitive material (such as aSe).

aSe could capture incoming scintillation photons and provide an additional voltage measurement at each pixel.
Depending on the sensitivity, such a measurement could be used to reconstruct tracks by providing a $\frac{dE}{dX}$ measurement, or even be used as a time-tag or a trigger.

The use of a reference trigger could be useful to establish event-time within the same system, and allow adjacent pixels which would receive photons, but not charge, to contribute to time reconstruction.
Any reconstructed event requires some $T_{o}$ time to indicate the start of the event.
Typically this is done via scintillation photons from a secondary system, where the photons arrive nearly instantly at the collection planes compared to the slow drift speed of the electrons.

The natural pixelization of Q-Pix required the charge collection can also be used to be sensitive to scintillation photons.
These photons could not only provide the required event timing but also provide an additional means of calorimetry and track reconstruction.
Additional work is currently underway to demonstrate the viability.

Currently there are two competing geometries to incorporate light collection within Q-Pix.
Figure~\ref{fig:qpix_light_geometries} shows the vertical and horizontal drift geometries.
The vertical drift geometry is an easier design geometry, however the top electrode must not only be able to collect charge it must also be transparent. 
On the other hand, the horizontal drift geometry solves the transparency problem but involves a more complicated hardware design. 

\begin{figure}[]
\centering
\begin{subfigure}{.45\textwidth}
  \centering
  \includegraphics[width=\textwidth]{images/qpix_light_vertical_geom.pdf}
  \caption{Vertical Drift}
\end{subfigure}%
\begin{subfigure}{.45\textwidth}
  \centering
  \includegraphics[width=\textwidth]{images/qpix_light_horizontal_geom.pdf}
  \caption{Horizontal Drift}
\end{subfigure}
\caption{Images taken from~\citep{https://doi.org/10.48550/arxiv.2207.11127}.
The vertical drift geometry (left) shows an electrode on the top layer, whereas the horizontal (right) geometry integrates the electrodes within the aSe.
The vertical geometry is an easier design, but most electrodes are not transparent to VUV scintillation light.
The horizontal geometry solves this problem, at the cost of a more complicated design.
}
\label{fig:qpix_light_geometries}
\end{figure}

\section{Q-Pix at Low Energy: Supernova Studies}
\label{sec:qpix_supernova}
Work has been done to characterize a Q-Pix readout ability measure core collapse supernovae~\citep{qpix:shion} events within a DUNE-FD module.
These studies involved particle Geant4-based(~\citep{geant4:AGOSTINELLI2003250}) simulations for low energy ($\simeq$ 10~\unit{MeV}) neutrino events.
The results indicate several advantages Q-Pix readout will have over a traditional wire-based readout.

One advantage is the lower overall data rates (1.03$*10^{-6}$ < 2~\unit{pB} per year) compared to the traditional single-phase readout.
This reduction in data rates could allow for a 10 kT Q-Pix module to collect timestamp data continually.
Such a continuous readout, with no trigger, could be particularly useful in collecting supernova burst events.
Figure~\ref{fig:qpix_shion} compares the trigger sensitivity supernova trigger burst efficiencies between the Q-Pix readout against traditional wire-readout. 

\begin{figure}[]
\centering
\includegraphics[width=0.6\textwidth]{images/shion_qpix_snb_trigger.jpg}
\caption{Image is taken directly from~\citep{qpix:shion}.
Plotted is the supernova burst triggering efficiency as a function of $\nu_{e}$ interactions in a 10 kT DUNE-FD module.
The points in blue indicate that a series of 60 resets are "clustered" to use as an identification of a supernova event. 
The other curves are taken from Ref.~\citep{supernova_Abi_2021}.
}
\label{fig:qpix_shion}
\end{figure}

%% chapter 3 - cap everything off with extended simulation studies based on results
In this chapter, we present the first implementation of the Q-Pix-based design using off-the-shelf electronics.

This section describes the first prototype based on the Q-Pix readout: The Simplified Analog Q-Pix (SAQ).
First we discuss the design goals of the prototype and highlight the basic building blocks of any Q-Pix based prototype.
Next, We describe the prototype status as well as lessons learned in characterizing noise and performing calibrations.

In the final part of this section we describe the future goals of this prototype, including the addition of GEMs to the experimental setup.
The full results of the planned diffusion measurements are beyond the scope of this work, but we provide the initial details here because these measurements will ultimately provide the complete description of the prototype.

\section{Simplified Analog Q-Pix: System Design}

The SAQ prototype is designed as a first physical proof-of-concept for a Q-Pix readout.
The intended use

\section{The SAQ Protoype Design}

\begin{figure}[]
\centering
\includegraphics[width=\textwidth]{images/SAQ_16_ivc_readout_board.pdf}
\caption{The SAQ Setup model based on~\ref{}.}
\end{figure}~\label{fig:saq_readout_board}

%%
\begin{figure}[]
\centering
\includegraphics[width=\textwidth]{images/SAQ_physical_setup.jpg}
\caption{The SAQ Setup model based on~\ref{fig:saq_setup_physical}.}
\end{figure}~\label{fig:saq_setup_flatten}

\begin{figure}[]
\centering
\includegraphics[width=\textwidth]{images/SAQ_setup_diagram.pdf}
\caption{The SAQ Setup model based on~\ref{fig:saq_setup_diagram}.}
\end{figure}~\label{fig:saq_setup_flatten}

\subsection{The TPC Design}
%% closeup image of the TPC here

\subsection{The Integrator Circuit}



\begin{figure}[]
\centering
\includegraphics[width=\textwidth]{images/SAQ_spice_circuit.pdf}
\caption{The SAQ circuit in a Spice Simulation. The IVC~\citep{ivc_datasheet} chip chosen as the off-the-shelf integrator for this experiment. The main selection choice for this part is due to its low input bias current $\ll 750~\unit{fA}$.}
\end{figure}~\label{fig:saq_circuit_spice}


\subsection{The SAQ Data Acquisition}

All resets are recorded via a Zybo-Z7-20 Digilent FPGA prototype board, which uses an Artix Zynq based archticture.
The reference manual for the Zybo Z7 board used in SAQ can be found at \citep{zybo_zy_reference}.

\begin{figure}[]
\centering
\includegraphics[width=\textwidth]{images/SAQ_zybo_daq.pdf}
\caption{An image of the data acquisition board from Digilent, Zybo Z7-20. This board was chosen for its multiple configurable input chanels, as well as the Zynq-based archiecture of the onboard FPGA. Additionally, the use of the ethernet provides $1~\unit{GB}$ transfer speeds, which is more than sufficient for the application.}
\end{figure}~\label{fig:saq_zybo}

\begin{figure}[]
\centering
\includegraphics[width=\textwidth]{images/SAQ_gui_resets.pdf}
\caption{The SAQ GUI with real time plotting of incoming resets to the Zybo board.}
\end{figure}~\label{fig:saq_gui}


\section{Noise Measurements}

The Q-Pix readout is dependent on the integrator, which provides the basic datum of the reset time.
Therefore, a dominant source of noise are electrons which accumulate on the integrator which are not signal electrons.
There are two possible sources for these noise electrons: excess electrons produced from the target volume or leakage current due to transistor effects from the integrator circuit.
In this section we focus on the noise electrons due to the leakage current.

\subsection{Integrating towards background Current}

Leakage current arrises due to non-idyllic behavior of the integrator operational amplifier, where the voltage across the two input terminals is nonzero.
Measurements of this leakage current then are performed by measuring voltage difference across the terminals as well as directly using a pico-ammeter.

\subsection{Integrating towards background Current}

%% describe setup / filling of TPC here
The second source of noise electrons are produced from the target volume.
The target volume is an ultra pure Argon Gas at TODO militorr.
% 14 psi with argon (0.069 bar)
% And ~3mtorr of vacuum before that
In this case the excess electrons come from the nominal decay of Ar-39, which provide excess electrons from the natural $\beta$ decay, at a rate of $\approx 1~\unit{Bq}{Kg^{-1}}$

\subsection{Digital Noise Sources and Clock Stability}


\section{Xenon Gas Lamp Measurements}

\begin{figure}[]
\centering
\includegraphics[width=\textwidth]{images/SAQ_gui_resets.pdf}
\caption{The SAQ GUI with real time plotting of incoming resets to the Zybo board.}
\end{figure}~\label{fig:saq_gui}

\section{Results and Discussion}

\begin{figure}[]
\centering
\includegraphics[width=\textwidth]{images/SAQ_first_diffusion_measurement.pdf}
\caption{First diffusion measurement in P-10 gas performed at Wellesy University.}
\end{figure}~\label{fig:saq_first_diffusion_measurement}

\subsection{Current Status and Planned Measurements}

Measurements of Transverse and Longitudinal diffusion of electrons within electric fields of strength 500 V/cm have been performed before \citep{lar_diffusion_measurement_LI2016160}.


%% chapter 4
\section{Digital Design Overview, System Requirements}

\section{A Review of the FSM outline}

\section{The Parameter Space of a Digital System.}

\section{Simulation Studies}

\subsection{Buffer Depth Requirements}

%% QDB Hardware Discussion
\section{QDB Design Overview}

\section{Power and Current Characteristics}

\section{Timing Stability}

\section{Analysis of Systematics for Different System Implementations}

\section{Summary}

%% chapter 5


\section{Physical Simulation Studies}

\section{Background Rates and Calibration}

sources of backgrounds are taken from \citep{DUNE-FD_TDRv4:Abi_2020}

\section{Supernova Studies}

Work has been done to understand how a Q-Pix based DUNE-FD would measure core collapse supernovae \citep{qpix:shion}.

Simulation studies which involved particle interactions were based on Geant4 \citep{geant4:AGOSTINELLI2003250}.


\section{Looking for Hadron Decay}

\section{Neutrino Beam High Energy Studies}

\section{Further Studies}

%% chapter 6 - Summary and Outlook
\chapter{Summary and Outlook}~\label{chap:summary}

\section{Conclusions}

The results presented in this work provide the first tests and verification of the digital back-end for novel pixel readout technology targeted at liquid Argon Time-Project-Chambers.

The first Q-Pix analog prototype using Off-The-Shelf analog components has been built and is currently taking measurements.
This prototype promises to provide gaseous Argon diffusion measurements, which will likely be the first true physics measurement using a Q-Pix based readout.

We have built and verified the first digital prototype boards which have verified communication reliability to protect against potential data loss.
We developed a frequency calibration method for remote nodes to demonstrate Q-Pix's ability to have independent oscillators.
We used this prototype and verified the ability to reconstruct remote oscillator frequencies with a precision more than an order of magnitude required (0.1~\unit{ppm} < 1~\unit{ppm}).
These results are verified between two different interrogation frequencies.

We developed multiple simulations to model the detector's response to long ($1000 second$) run time exposure of radiogenic backgrounds as well as tested the ability to readout beam neutrino events at LBNF.
Our simulations show that the current ASIC's local (64) and remote (128) FIFO depths of the digital ASIC prototype are too small to \unit{GeV} scale neutrino events in a DUNE-FD.
We estimate that the local FIFO depth should be at least be able to record 426 unique resets in order to fully capture 99\% of neutrino events with energy up to 10~\unit{GeV}.
These results are modified to a required 394 when accounted for expected (dis)appearance spectra given in~\citep{DUNE_FD_TDRv2_2020}
This result provides the first limit on the memory required for a Q-Pix ASIC, should it be used in a DUNE-FD module to measure neutrino oscillations.

To test the remote FIFO depths and local oscillator frequency requirements we developed the first simulation to model the Q-Pix digital back-end response to physical events within a DUNE-FD APA.
We find that the distribution of the ASIC frequency needs to be $\approx 0.5\%$ in order to maintain obtain reliable remote FIFO depths with the current readout protocol. 
These results also indicate that the only reliable routing methodolgy is the "Snake" routing (Section~\ref{sec:snake_timing}), which is shown to be independent of both tile size and digital architecture (See Table~\ref{table:tile_params}).
The routing ("Snake") provides a unitary relationship between the local and remote FIFO depth requirements.

\subsection{The Future of Q-Pix}

The Q-Pix design is a novel readout technology.
However, "novelty does not confer automatically benefit", David Nygren.
The full Q-Pix validation still awaits key results to demonstate its capabilities in a DUNE-FD module.
Namely, Q-Pix still needs to test both the analog and digital prototypes at cold liquid Argon temperatures.

The front-end requires a reliable replenishment circuit as well as low leakage current ($\approx 100~\unit{aA}$ or less to be below radiogenic backgrounds).
Also, when known, the timing response of the replenishment circuit should be applied to the RTD results presented in this work.
Knowledge of the timing response of the analog front-end can be combined with the neutrino simulation events shown here to allow for accurate event reconstruction.
These reconstructed events will permit an analysis to estimate of Q-Pix's ability to perform neutrino oscillation measurements.

\subsection{Q-Pix's First and Second Digital Prototypes}

The work presented here can accurately be viewed both as a means to understand the Q-Pix's first digital ASIC and as a guide toward the second digital design.
The key result of this work indicates that the local and remote FIFO depths of the second prototype should both be increased to at or above 394 to capture 99\% of neutrino oscillation interactions.
The reason the first prototype did not incorporate these larger buff sizes was due to fabrication limitiations of the ASIC.
If oscillator tests of the first prototype indicate that the mean drift between neighbor ASICs is reliably under 0.5\%, then the local oscillator need not be changed either.
All other underlying logic, with perhaps the exception of FWFT FIFOs, have been verified in the first digital prototype.
These tests need only be repeated on the first prototype ASIC.

Eventually the Q-Pix front and back-end ASICs will likely be combined into a single chip.
Still, the motivation provided by the results presented here for the second prototype (applied only to the digital portion) remain unchanged.


\printbibliography[heading=bibintoc]
% \printbibliography

%% appendix work on SVSC?
\appendix

% \chapter{SVSC OS1}
% \label{chap:OS1}
% the work in this subsection details the work and results of \cite{svsc_os1_aline_2021}. 

% \chapter{SVSC OS2}
% \label{chap:OS2}
% Put OS2 work here
% The section lists the work detailed in \citep{svsc_os2_Keefe_2022}.


\chapter{Neutrino Interaction Integral Data}~\label{app:integral_data}
\begin{table}
	\begin{center}
		\begin{tabular}{|l|l|l|l|l|l|}
			\hline
			lepton Pdg & Horn Current Direction & Z-Pos & Theta & 95\% Capture & 99\% Capture \\
			\hline
			12 & forward & 10 & 0 & 8575 & 9575 \\
			\hline
			12 & forward & 80 & 0 & 8775 & 9675 \\
			\hline
			12 & forward & 180 & 0 & 8775 & 9725 \\
			\hline
			12 & forward & 280 & 0 & 8675 & 9725 \\
			\hline
			12 & forward & 350 & 0 & 8075 & 9575 \\
			\hline
			12 & forward & 10 & 0 & 694 & 930 \\
			\hline
			12 & forward & 80 & 0 & 818 & 966 \\
			\hline
			12 & forward & 180 & 0 & 846 & 970 \\
			\hline
			12 & forward & 280 & 0 & 854 & 990 \\
			\hline
			12 & forward & 350 & 0 & 770 & 966 \\
			\hline
			12 & forward & 10 & 2 & 8375 & 9625 \\
			\hline
			12 & forward & 80 & 2 & 8975 & 9725 \\
			\hline
			12 & forward & 180 & 2 & 8775 & 9725 \\
			\hline
			12 & forward & 280 & 2 & 8725 & 9675 \\
			\hline
			12 & forward & 350 & 2 & 8075 & 9575 \\
			\hline
			12 & forward & 10 & 2 & 714 & 922 \\
			\hline
			12 & forward & 80 & 2 & 814 & 978 \\
			\hline
			12 & forward & 180 & 2 & 850 & 986 \\
			\hline
			12 & forward & 280 & 2 & 834 & 970 \\
			\hline
			12 & forward & 350 & 2 & 698 & 938 \\
			\hline
			12 & forward & 10 & -2 & 8425 & 9475 \\
			\hline
			12 & forward & 80 & -2 & 8875 & 9725 \\
			\hline
			12 & forward & 180 & -2 & 8825 & 9675 \\
			\hline
			12 & forward & 280 & -2 & 8775 & 9725 \\
			\hline
			12 & forward & 350 & -2 & 8125 & 9375 \\
			\hline
			12 & forward & 10 & -2 & 694 & 942 \\
			\hline
			12 & forward & 80 & -2 & 834 & 986 \\
			\hline
			12 & forward & 180 & -2 & 846 & 982 \\
			\hline
			12 & forward & 280 & -2 & 834 & 982 \\
			\hline
			12 & forward & 350 & -2 & 786 & 978 \\
			\hline
			12 & forward & 10 & 90 & 8625 & 9675 \\
			\hline
			12 & forward & 80 & 90 & 8575 & 9625 \\
			\hline
			12 & forward & 180 & 90 & 8675 & 9675 \\
			\hline
			12 & forward & 280 & 90 & 8725 & 9675 \\
			\hline
			12 & forward & 350 & 90 & 4625 & 8725 \\
			\hline
			12 & forward & 10 & 90 & 874 & 990 \\
			\hline
			12 & forward & 80 & 90 & 874 & 986 \\
			\hline
			12 & forward & 180 & 90 & 850 & 974 \\
			\hline
			12 & forward & 280 & 90 & 858 & 974 \\
			\hline
			12 & forward & 350 & 90 & 530 & 886 \\
			\hline
			12 & forward & 10 & -90 & 3925 & 7375 \\
			\hline
			12 & forward & 80 & -90 & 8775 & 9675 \\
			\hline
			12 & forward & 180 & -90 & 8725 & 9625 \\
			\hline
			12 & forward & 280 & -90 & 8675 & 9725 \\
			\hline
			12 & forward & 350 & -90 & 8625 & 9775 \\
			\hline
			12 & forward & 10 & -90 & 474 & 798 \\
			\hline
			12 & forward & 80 & -90 & 842 & 978 \\
			\hline
			12 & forward & 180 & -90 & 878 & 990 \\
			\hline
			12 & forward & 280 & -90 & 878 & 986 \\
			\hline
			12 & forward & 350 & -90 & 886 & 986 \\
			\hline
			12 & forward & 10 & 0 & 8575 & 9575 \\
			\hline
			12 & forward & 10 & 2 & 8375 & 9625 \\
			\hline
			12 & forward & 10 & -2 & 8425 & 9475 \\
			\hline
			12 & forward & 10 & 90 & 8625 & 9675 \\
			\hline
			12 & forward & 10 & -90 & 3925 & 7375 \\
			\hline
			12 & forward & 10 & 0 & 694 & 930 \\
			\hline
			12 & forward & 10 & 2 & 714 & 922 \\
			\hline
			12 & forward & 10 & -2 & 694 & 942 \\
			\hline
			12 & forward & 10 & 90 & 874 & 990 \\
			\hline
			12 & forward & 10 & -90 & 474 & 798 \\
			\hline
			12 & forward & 80 & 0 & 8775 & 9675 \\
			\hline
			12 & forward & 80 & 2 & 8975 & 9725 \\
			\hline
			12 & forward & 80 & -2 & 8875 & 9725 \\
			\hline
			12 & forward & 80 & 90 & 8575 & 9625 \\
			\hline
			12 & forward & 80 & -90 & 8775 & 9675 \\
			\hline
			12 & forward & 80 & 0 & 818 & 966 \\
			\hline
			12 & forward & 80 & 2 & 814 & 978 \\
			\hline
			12 & forward & 80 & -2 & 834 & 986 \\
			\hline
			12 & forward & 80 & 90 & 874 & 986 \\
			\hline
			12 & forward & 80 & -90 & 842 & 978 \\
			\hline
			12 & forward & 180 & 0 & 8775 & 9725 \\
			\hline
			12 & forward & 180 & 2 & 8775 & 9725 \\
			\hline
			12 & forward & 180 & -2 & 8825 & 9675 \\
			\hline
			12 & forward & 180 & 90 & 8675 & 9675 \\
			\hline
			12 & forward & 180 & -90 & 8725 & 9625 \\
			\hline
			12 & forward & 180 & 0 & 846 & 970 \\
			\hline
			12 & forward & 180 & 2 & 850 & 986 \\
			\hline
			12 & forward & 180 & -2 & 846 & 982 \\
			\hline
			12 & forward & 180 & 90 & 850 & 974 \\
			\hline
			12 & forward & 180 & -90 & 878 & 990 \\
			\hline
			12 & forward & 280 & 0 & 8675 & 9725 \\
			\hline
			12 & forward & 280 & 2 & 8725 & 9675 \\
			\hline
			12 & forward & 280 & -2 & 8775 & 9725 \\
			\hline
			12 & forward & 280 & 90 & 8725 & 9675 \\
			\hline
			12 & forward & 280 & -90 & 8675 & 9725 \\
			\hline
			12 & forward & 280 & 0 & 854 & 990 \\
			\hline
			12 & forward & 280 & 2 & 834 & 970 \\
			\hline
			12 & forward & 280 & -2 & 834 & 982 \\
			\hline
			12 & forward & 280 & 90 & 858 & 974 \\
			\hline
			12 & forward & 280 & -90 & 878 & 986 \\
			\hline
			12 & forward & 350 & 0 & 8075 & 9575 \\
			\hline
			12 & forward & 350 & 2 & 8075 & 9575 \\
			\hline
			12 & forward & 350 & -2 & 8125 & 9375 \\
			\hline
			12 & forward & 350 & 90 & 4625 & 8725 \\
			\hline
			12 & forward & 350 & -90 & 8625 & 9775 \\
			\hline
			12 & forward & 350 & 0 & 770 & 966 \\
			\hline
			12 & forward & 350 & 2 & 698 & 938 \\
			\hline
			12 & forward & 350 & -2 & 786 & 978 \\
			\hline
			12 & forward & 350 & 90 & 530 & 886 \\
			\hline
			12 & forward & 350 & -90 & 886 & 986 \\
			\hline
		\end{tabular}
	\end{center}
	\caption{Integral Data}
	\label{tab:sum}
\end{table}

\chapter{Q-Pix Analog ASIC First Prototype}~\label{app:analog_prototype}
\begin{figure}[]
\centering
\includegraphics[width=\textwidth]{images/upcoming_qpix_ASIC.pdf}
\caption{Full Schematic of the upcoming Q-Pix analog front-end ASIC.}
\end{figure}

\printindex

\end{document}
